\documentclass[12pt]{article}
%\documentclass{nature}

% Including pdf figures
\usepackage{graphicx}
\graphicspath{ {../Plots/} }
\usepackage{pdfpages}
%really place a figure in a location
\usepackage{float}
%Overrun caption
\usepackage[CaptionAfterwards]{fltpage}
% Math stuff
\usepackage{amsmath}
\usepackage{stix}
% Bibliographies
\usepackage[numbers]{natbib}
\bibpunct{(}{)}{,}{a}{}{;} 

\usepackage{hyperref}

\usepackage[flushleft]{threeparttable}

\usepackage[font={scriptsize}]{caption}

\usepackage{lineno} %gives line numbers with \lineno command

\usepackage{setspace}
\onehalfspace

\usepackage{tikz}%for putting words on figures
\usetikzlibrary{positioning}%for relative positioning

%to rotate figures
\usepackage{rotating}
\usepackage{pdflscape}

%external document referencing
\usepackage{xr}

\begin{document}

\externaldocument{Turnover_plos.tex}
\externaldocument{S1_Appendix.tex}
%\externaldocument{S2_Appendix.tex}
\externaldocument{S3_Appendix.tex}

\setcounter{equation}{0}
\renewcommand{\theequation}{S2.\arabic{equation}}

\section*{S2 Appendix}

\subsection*{Resident equilibria and stability}

In the resident population (allele $M$ fixed), we follow the frequency of $A$ in X-bearing female gametes, $p^\female_X$, and X-bearing male gametes, $p^\male_X$, and Y-bearing male gametes, $p^\male_Y$.
We also track the total frequency of Y among male gametes, $q$, which may deviate from $1/2$ due to meiotic drive in males. 
These four variables determine the frequencies of the six resident gamete types: $x_{1}^{\female}=\hat{p}_X^\female$, $x_{2}^{\female}=1-\hat{p}_X^\female$, $x_{1}^{\male}=(1-q)\hat{p}_X^\male$, $x_{2}^{\male}=(1-q)(1-\hat{p}_X^\male)$, $y_{1}^{\male}=q \hat{p}_Y^\male$, and $y_{2}^{\male}=q(1-\hat{p}_Y^\male)$. 
Mean fitnesses in the resident population are given in table \nameref{tab:meanfitnesses}.

Various forms of selection can maintain a polymorphism at the \textbf{A} locus, including sexually antagonistic selection, overdominance, conflicts between diploid selection and selection upon haploid genotypes \citep[ploidally antagonistic selection,][]{Immler:2012tl}, or a combination of these selective regimes (see below). 

In particular special cases, e.g., no sex-differences in selection or meiotic drive ($s^\male=s^\female$, $h^\male=h^\female$, and $\alpha^\male=\alpha^\female=1/2$), the equilibrium allele frequency and stability can be calculated analytically without assuming anything about the relative strengths of selection and recombination. 
However, here, we focus on two regimes (tight linkage and weak selection) in order to make fewer assumptions about fitnesses. 

\subsubsection*{Recombination weak relative to selection (tight linkage between \textbf{A} and \textbf{X})}

We first calculate the equilibrium frequency of the Y and $A$ alleles in the ancestral population when the recombination rate between the \textbf{X} and \textbf{A} loci is small ($r$ of order $\epsilon$). 
Selection at the \textbf{A} locus will not affect evolution at the novel sex-determining locus, \textbf{M}, if one allele is fixed on all backgrounds. 
We therefore focus on the five equilibria that maintain both $A$ and $a$ alleles, four of which are given to leading order by:

\begin{subequations}\label{eq:tightequil}
\begin{align}
(A)\ \ \ &\hat{p}_{Y}^{\male}=0,
\ \hat{q}=\frac{1}{2}\left(1 - \alpha_\Delta^\male \frac{ w_{Aa}^{\male} \phi}{w_{Aa}^{\male} \phi+ w_{aa}^{\male} \psi}\right),\\
&\hat{p}_{X}^{\female}=\frac{w_{a}^{\female} \phi}{w_{a}^{\female} \phi+ w_{A}^{\female} \psi},
\ \hat{p}_{X}^{\male}=\frac{(1+\alpha_\Delta^\male)w_{Aa}^{\male} \phi}{(1+\alpha_\Delta^\male)w_{Aa}^{\male} \phi +w_{aa}^{\male} \psi} \nonumber \\
(A')\ \ \ &\hat{p}_{Y}^{\male}=1,
\ \hat{q}=\frac{1}{2}\left(1 + \alpha_\Delta^\male \frac{w_{Aa}^{\male} \phi'}{w_{Aa}^{\male} \phi' + w_{AA}^{\male} \psi'}\right),\\
&\hat{p}_{X}^{\female}=1-\frac{w_{A}^{\female} \phi'}{w_{A}^{\female} \phi'+ w_{a}^{\female} \psi'},
\ \hat{p}_{X}^{\male}=1-\frac{(1-\alpha_\Delta^\male)w_{Aa}^{\male} \phi'}{(1-\alpha_\Delta^\male)w_{Aa}^{\male} \phi'+w_{AA}^{\male} \psi'} \nonumber \\
(B)\ \ \ &\hat{p}_{Y}^{\male}=0,\ \hat{p}_{X}^{\female}=1,\ \hat{p}_{X}^{\male}=1, \ \hat{q}=(1 -\alpha^\male_\Delta)/2\\
(B')\ \ \ &\hat{p}_{Y}^{\male}=1,\ \hat{p}_{X}^{\female}=0,\ \hat{p}_{X}^{\male}=0, \ \hat{q}=(1+\alpha^\male_\Delta)/2
\end{align}
\end{subequations}
\begin{equation*}
\begin{split}
\phi=&(1+\alpha^\female_\Delta) w_{A}^{\female} w_{Aa}^{\female} \left[ w_{a}^{\male} w_{aa}^{\male} + (1+\alpha_\Delta^\male) w_{A}^{\male} w_{Aa}^{\male} \right]/2 - w_{a}^{\male} w_{a}^{\female} w_{aa}^{\male} w_{aa}^{\female} \\
\psi=&(1-\alpha^\female_\Delta) w_{a}^{\female} w_{Aa}^{\female} \left[ w_{a}^{\male} w_{aa}^{\male} + (1+\alpha_\Delta^\male) w_{A}^{\male} w_{Aa}^{\male}\right]/2 - (1+\alpha_\Delta^\male) w_{A}^{\male} w_{A}^{\female} w_{Aa}^{\male} w_{AA}^{\female}\\
\phi'=&(1-\alpha^\female_\Delta) w_{a}^{\female} w_{Aa}^{\female} \left[ w_{A}^{\male} w_{AA}^{\male} + (1-\alpha_\Delta^\male) w_{a}^{\male} w_{Aa}^{\male} \right]/2 - w_{A}^{\male} w_{A}^{\female} w_{AA}^{\male} w_{AA}^{\female}\\
\psi'=&(1+\alpha^\female_\Delta) w_{A}^{\female} w_{Aa}^{\female} \left[ w_{A}^{\male} w_{AA}^{\male} + (1-\alpha_\Delta^\male) w_{a}^{\male} w_{Aa}^{\male} \right]/2 - (1-\alpha_\Delta^\male) w_{a}^{\male} w_{a}^{\female} w_{Aa}^{\male} w_{aa}^{\female}
\end{split}
\end{equation*}

\noindent
A fifth equilibrium $(C)$ also exists where $A$ is present at an intermediate frequency on the Y chromosome ($0<\hat{p}_{Y}^{\male}<1$). 
However, equilibrium $(C)$ is never locally stable when $r \approx 0$ and is therefore not considered further.
Thus, the Y can either be fixed for the $a$ allele (equilibria $A$ and $B$) or the $A$ allele (equilibria $A'$ and $B'$).
The X chromosome can then either be polymorphic (equilibria $A$ and $A'$) or fixed for the alternative allele (equilibria $B$ and $B'$).
Since equilibria $(A)$ and $(B)$ are equivalent to equilibria $(A')$ and $(B')$ with the labelling of $A$ and $a$ alleles interchanged, we discuss only equilibria $(A)$ and $(B)$, in which the Y is fixed for the $a$ allele. 
If there is no haploid selection ($\alpha^{\Hermaphrodite}_\Delta=0$, $w_{A}^{\Hermaphrodite}=w_{a}^{\Hermaphrodite}=1$), these equilibria are equivalent to those found by \cite{Lloyd1977} and \cite{Otto2014}.

We next calculate when $(A)$ and $(B)$ are locally stable for $r=0$. 
According to the `small parameter theory' \citep{Karlin:1972ab,Karlin:1972dq}, these stability properties are unaffected by small amounts of recombination between the SDR and \textbf{A} locus, although equilibrium frequencies may be slightly altered. 
For the $a$ allele to be stably fixed on the Y we need $\bar{w}_{Ya}^{\male} > \bar{w}_{YA}^{\male}$ where $\bar{w}_{Ya}^{\male} = w_{a}^{\male} \big[\hat{p}_X^\female(1-\alpha^\male_\Delta) w_{A}^{\female} w_{Aa}^{\male} + (1-\hat{p}_X^\female)w_{a}^{\female} w_{aa}^{\male} \big]$ and $\bar{w}_{YA}^{\male} = w_{A}^{\male} \big[ \hat{p}_X^\female w_{A}^{\female} w_{AA}^{\male} + (1-\hat{p}_X^\female)(1+\alpha^\male_\Delta) w_{a}^{\female} w_{Aa}^{\male} \big]$. 
That is, Y-$a$ haplotypes must have higher fitness than Y-$A$ haplotypes.  
Substituting in $\hat{p}_X^\female$ from equation \eqref{eq:tightequil}, fixation of the $a$ allele on the Y requires that $\gamma_{i}>0$ where $\gamma_{(A)}=w_{a}^{\male} \big[ (1-\alpha^\male_\Delta) w_{Aa}^{\male} \phi + w_{aa}^{\male} \psi \big]-w_{A}^{\male} \big[ w_{AA}^{\male} \phi + (1 + \alpha^{\male}_\Delta)w_{Aa}^{\male} \psi \big]$ for equilibrium $(A)$ and $\gamma_{(B)}=(1-\alpha^{\male}_\Delta)w_{a}^{\male}w_{Aa}^{\male}-w_{A}^{\male}w_{AA}^{\male}$ for equilibrium $(B)$.
Stability of a polymorphism on the X chromosome (equilibrium $A$) further requires that $\phi >0$ and $\psi >0$. 
Fixation of the $a$ allele on the X (equilibrium $B$) can be stable only if equilibrium $(A)$ is not, as it requires $\psi<0$. %and $2w_{A}^{\female}w_{AA}^{\female}>(1-\alpha_\Delta^{\female})w_{a}^{\female}w_{Aa}^{\female}$ or just $4w_{A}^{\female}w_{AA}^{\female}<(1-\alpha_\Delta^{\female})w_{a}^{\female}w_{Aa}^{\female}$ (which also prevents $\psi>0$). 

%\textcolor{red}{check last condition and the stability condition below are correct}
%\textcolor{blue}{The last condition looks good to me, although in your Turnover-norec-MFS.nb you look at YA fixed, so you have to flip everything (so I made Turnover-norec-MFS-MMO.nb to do this).
%The one issue I can find here is that you can also prevent $\lambda>1$ when the slope and intercept of the quadratic at $\lambda=1$ are negative (you only looked at both being positive). In this case we need $4w_{A}^{\female}w_{AA}^{\female}<(1-\alpha_\Delta^{\female})w_{a}^{\female}w_{Aa}^{\female}$, which also prevents $\psi>0$.
%I've added this in.
%It could also be the case that the slope and intercept are the same sign but the roots are imaginary - but this is never the case here.
%Stability condition below looks good to me (from matt version of turnoverSOM-MIKE.nb).}

\subsubsection*{Selection weak relative to recombination (weak selection)}

Here, we assume that selection and meiotic drive are weak relative to recombination ($s^\Hermaphrodite$, $t^\Hermaphrodite$, $\alpha_{\Delta}^\Hermaphrodite$ of order $\epsilon$). 
The maintenance of a polymorphism at the \textbf{A} locus then requires that

\begin{equation}
\begin{split}
0&< - \left[(1-h^\female)s^\female +(1-h^\male) s^\male + t^\female +t^\male + \alpha_{\Delta}^\female+\alpha_{\Delta}^\male\right]\\
%
\text{and }\quad 0&<h^\female s^\female +h^\male s^\male + t^\female +t^\male + \alpha_{\Delta}^\female+\alpha_{\Delta}^\male.
\end{split}
\end{equation}

\noindent
which indicates that a polymorphism can be maintained by various selective regimes. 

Given that a polymorphism is maintained at the \textbf{A} locus by weak selection, the frequencies of $A$ in each type of gamete are the same ($\hat{p}^\female_X=\hat{p}^\male_X=\hat{p}^\male_Y=\bar{p}$) and given, to leading order, by 

\begin{equation}
\bar{p}=\frac{h^\female s^\female + h^\male s^\male +t^\female+t^\male+\alpha_{\Delta}^\female+\alpha_{\Delta}^\male}
{(2h^\female-1)s^\female+(2h^\male-1)s^\male}
+O(\epsilon)
.
\label{eq:pAve}
\end{equation}

\noindent
Differences in frequency between gamete types are of $O(\epsilon)$:

\begin{equation}
\begin{split}
\hat{p}^\male_X-\hat{p}^\female_X&=V_{A}\big{(}D^\male - D^\female +\alpha_{\Delta}^\male-\alpha_{\Delta}^\female \big{)}
+O(\epsilon^2)\\
%
\hat{p}^\male_Y-\hat{p}^\female_X&=V_{A}\left[D^\male - D^\female+\alpha_{\Delta}^\male-\alpha_{\Delta}^\female+(1-2r)(t^\male-t^\female)\right]/2r
+O(\epsilon^2)\\
%
\hat{p}^\male_Y-\hat{p}^\male_X&=V_{A}\left(D^\male - D^\female+\alpha_{\Delta}^\male-\alpha_{\Delta}^\female+t^\male-t^\female\right)(1-2r)/2r
+O(\epsilon^2)
\end{split}
\label{eq:freq_diffs}
\end{equation}

\noindent
where $V_{A}=\bar{p}(1-\bar{p})$ is the variance in the frequency of $A$ and $D^\Hermaphrodite=\big{[} \bar{p}s^\Hermaphrodite+(1-\bar{p})h^\Hermaphrodite s^\Hermaphrodite\big{]} -\big{[} \bar{p}h^\Hermaphrodite s^\Hermaphrodite+(1-\bar{p}) \big{]}$ corresponds to the difference in fitness between $A$ and $a$ alleles in diploids of sex $\Hermaphrodite \in \{\female,\male\}$ ($\bar{p}$ is the leading-order probability of mating with an $A$-bearing gamete from the opposite sex). 
The frequency of Y among male gametes depends upon the difference in the frequency of the $A$ allele between X- and Y-bearing male gametes and the strength of meiotic drive in favour of the $A$ allele in males, $q=1/2+\alpha_{\Delta}^\male(\hat{p}^\male_Y-\hat{p}^\male_X)/2+O(\epsilon^3)$.
Without gametic competition or drive ($\alpha_{\Delta}^\Hermaphrodite=t^\Hermaphrodite=0$) our results reduce to those of \citet{vanDoorn:2007eu}.

\bibliographystyle{amnatnat.bst}
\bibliography{sex_chromosomes.bib}

\end{document}



