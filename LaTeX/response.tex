\documentclass[10pt,letterpaper]{article}
\usepackage[top=0.85in,left=0.75in,right=0.75in,footskip=0.75in]{geometry}
\usepackage[utf8x]{inputenc}
\usepackage{amsmath,amssymb}
\bibliographystyle{plos2015}
\usepackage{color}

\begin{document}

\noindent\textbf{Color code}:\\
reviewers' words\\
\textcolor{red}{to do/thoughts/questions between coauthors (temporary)}\\
\textcolor{blue}{authors' response}

\section{Reviewer 1}

The turnover within a species from one genetic means of sex determination to another is surprisingly common, and it poses a challenging puzzle for evolutionary geneticists.  Existing theory on the topic suggests that there must be either sexually-antagonistic selection or selection variants that bring the sex ratio closer to 1:1.  This paper builds explicit models of the situation and asks about the role of selection in the haploid stage (i.e. meiotic drive and gametic selection), and finds that such effects can also serve as the primary driver of turnover of sex determining mechanisms.  While this is a rather focused topic, it is a striking evolutionary puzzle, and the paper presents a very satisfying solution that is much more general than we had previously thought.  It also overturns the primacy of Fisherian sex ratio evolution as the driver of the dynamics of genes of this sort.

The authors devise a satisfyingly simple genetic scenario, with one ancestral sex-determining locus, one locus under selection, and a third locus for the nascent sex determining locus.  They manage to collapse some messy algebra into terms like the ``haplotypic growth rate" that allow the reader to intuit many of the results.  I only have a few suggestions:

\noindent\subsection{}
The selected locus allows for three modes of selection: gametic selection (weighting the frequencies of gamete genotypes), the usual diploid viability selection, and meiotic drive (where heterozygotes may produce gametes that deviate from 50:50 proportions).  It should be noted that the model does not accommodate fertility or sexual selection.

\noindent\paragraph{Response}
\textcolor{red}{One could interpret our diploid selection as fertility selection, or does the reviewer mean that we have only one selective event in the diploid phase? Sexual selection does seem like a different model though, with for example, male fitnesses depending on female frequencies. A note such as "We do not consider fertility or sexual selection." in the paragraph beginning near L111.}

\noindent\subsection{}
The ``haplotypic growth rate" plays a really essential role in providing intuition about the results. Similarly the $\chi_{mi}^{(XY)}$ terms also are crucial.  I suggest giving a short name for the chi terms (currently they are the ``rate at which mutant haplotypes on background i recombine onto the other A locus background in heterozygotes"...).   The importance of these becomes clear in Table 2, but I wonder if there is a graphical way to make their meaning even clearer.  

\noindent\paragraph{Response}
\textcolor{red}{We could call it the ``haplotypic recombination rate" near L153, but I'm not sure how much we will use this term in the text.}

\noindent\subsection{}
It needs to be stated somewhere that the models are entirely deterministic, implying that the authors assume that the selective forces always exceed the force of random drift. It might be useful to discuss this a bit. Is there evidence that in no case has drift been a key factor in changes of sex determination system?  The discussion mentions alleles of weaker effect, but does not mention that in this scenario, models of weak effects would need to consider drift as well.

\noindent\paragraph{Response}
\textcolor{red}{We could first mention that the model is deterministic, and hence ignores drift, near L131. As for the role of drift in transitions, we could certainly elaborate on the Veller et al 2017 Genetics paper that discusses how ``drift-induced selection" causes trans-GSD transitions. I've put a possible sentence in near L482: ``It should be noted, however, that the dynamics of sex-determining alleles with very weak effect will be influenced by genetic drift, which itself has been shown to cause trans-GSD transitions (Veller et al 2017)."}

\noindent\subsection{}
A glitch appears in the section ``Loose linkage with the ancestral sex-determining region" on line 270, where the text reads, ``To leading order in selection, the leading eigenvalues are..."

\noindent\paragraph{Response}
\textcolor{red}{Perhaps we could just drop the words ``the leading eigenvalues are" (which I have done)?}

\noindent\subsection{}
Gametic selection among sperm is a bit different from sperm competition, since the latter involves pair-wise (or multi-way) bouts, whereas gametic selection assigns constant fitnesses to each haplotype. I would replace occurrences of ``sperm competition" with ``male gametic selection", or maybe ``sperm gametic competition". To be consistent, replace ``egg competition" with ``female/egg gametic competition". 

\noindent\paragraph{Response}
\textcolor{blue}{We have made the suggested replacements. There are two exceptions. In the abstract we use sperm competition as an example of gametic competition, but we now state that this would involve multiple males. In the introduction we mention that sperm competition has been shown to alter allele frequencies and fitness, and in this case we now mention this was shown within a single ejaculate (L68).}

\noindent\subsection{}
I am not certain of this, but it seems that a potentially important difference between ESD and GSD has to do with the variance in proportion male? Since ESD (in a constant environment, as is implicitly modeled here) has a binomial sampling of sexes, but GSD might deviate from this.  This might be an issue because Fisherian sex ratio selection acts on the mean and not the variance.

\noindent\paragraph{Response}
\textcolor{red}{I think that, with our deterministic model (where the number of binomial trials is infinite), the variance in the fraction male is 0, and so is the same as GSD. We could briefly mention this at the beginning or end of the ESD section, and say that modeling ESD in a finite population could be interesting?}

\noindent\subsection{}
Full sharing of the Mathematica file is to be commended.

\section{Reviewer 2}

Scott et al. present a 3-locus model for sex-determination (SD) evolution (1 locus has an ancestral SD function; 1 has a derived SD function; 1 locus affects selection in the haploid or diploid phase of the life cycle). Their results converge on previously-identified conditions for the spread of new SD alleles, while also broadening the range of scenarios and conditions that can favour transitions between SD systems.

The paper is interesting and conceptually appropriate for PLoS Biology. The authors have done a great job of analysing a somewhat complicated model and drawing out a relatively complete picture of the conditions favouring the evolution of new sex determination systems, and I expect their expert treatment will go over well with people that are close to the research context of the new theory.

On the other hand, I am much less sure that this paper will reach beyond experts in this subject area, though the authors should seek to broaden the appeal. This is not because the results are narrow -- I think they can have a broad appeal. Rather, the paper is written in a technical manner that is likely to limit the appeal beyond the expert class. I am reasonably comfortable with models like this, but still found the presentation to be very hard going in a variety of places, particularly in the results. I offer some suggestions for presentation that may help improve the presentation and broaden the potential appeal of the paper.

\noindent\subsection{}
I think the introduction could be simplified so that the context of the study is more obvious. The authors essentially need to make three key points, which are there, but could be better drawn out. First, transitions between sex-determining systems are common, and recent empirical work has given us a much greater appreciation for just how common these transitions are. Second, there are two major mechanisms that are thought induce evolution of new sex-determination mechanisms: (1) selection to stabilize the sex ratio, and (2) hitchhiking between (tightly linked) sex-determination and sexually antagonistic alleles. Third, the authors should clearly outline the major limitations of current theory as a means of motivating their mega-model that captures an array of genetic and selection scenarios that may lead to transitions in sex determination. As I mentioned, these elements can all be found in the introduction, but the pitch can be substantially streamlined to allow readers to easily see: (1) what is the context? (2) What do we currently think drives transitions between SD systems? And (3) what are the holes in the theory and how will the authors address them?

\noindent\paragraph{Response}
\textcolor{blue}{We now ...}

\noindent\subsection{}
Likewise, the results yield three major insights:\\
(1) SD transitions can, counterintuitively, potentially INCREASE sex ratio bias\\
(2) Tight linkage and sexual antagonism are not required for transitions to occur\\
(3) Under haploid selection, conditions for the spread of new sex-determining alleles are extremely permissive.\\
Again, all of these points are there, but they could be better drawn out than they currently are. The results, as written, are hard-going, and I expect will be particularly challenging for readers that are not heavily immersed in evolutionary theory.

One potential way to help readers ? particularly those interested in what the paper has to say, but are not mathematically inclined ? is to conspicuously highlight each important result and briefly place it into context of earlier theory. But at the same time, the authors have done a great job of drawing out general mathematical results from a relatively complicated model ? their mathematical insights ARE the results and I am not suggesting that they be minimized or shunted into appendices.

One tactic for including analytical details which specialists will want to see, while allowing less mathematically inclined readers to easily find each main result, is to place short and accessible summaries at the close of each section of the Results. Such a tactic has been put to good use in a hugely influential paper by Allen Orr (1998 Evolution), which I expect the authors will be familiar with. Orr presents the major mathematical results of his analysis alongside prominently labelled punchline paragraphs that summarize each of his main results in a concise and easily understandable manner. If the authors choose to take this approach, please be clear to place the take-home messages in context: clearly state whether the result confirms previous theory, or represents a new result or contradiction of an earlier result.

\noindent\paragraph{Response}
\textcolor{blue}{We now ...}

\subsection{MINOR COMMENTS}

\noindent\subsubsection{}
L3: "of diploid sexes" can be deleted

\noindent\paragraph{Response}
\textcolor{blue}{We now ...}

\noindent\subsubsection{}
L12: missing a ``)"

\noindent\paragraph{Response}
\textcolor{blue}{We now ...}

\noindent\subsubsection{}
L35-37: sentence could use an edit for ease of reading.

\noindent\paragraph{Response}
\textcolor{blue}{We now ...}

\noindent\subsubsection{}
Immediately preceding the paragraph beginning line 44: It would be worth establishing the limitations of previous work. What exactly motivated you to do this mathematical modelling?

\noindent\paragraph{Response}
\textcolor{blue}{We now ...}

\noindent\subsubsection{}
L56-57: Rephrase for clarity "individual cases of meiotic drive are generally sex-limited, and exclusively effect male or female gametes". And on line 57, "sex specific" could be changed to "sex-limited".

\noindent\paragraph{Response}
\textcolor{blue}{We now ...}

\noindent\subsubsection{}
Lines 80-81: This primary finding is completely aligned with the earlier views of theory that are summarized above. Are you sure you want to lead with this as the main emphasis of the study? It understates what is new.

\noindent\paragraph{Response}
\textcolor{blue}{We now ...}

\noindent\subsubsection{}
M locus is dominant to X. How important is this assumption? Is this compatible with any data (e.g., on feminizing or masculinizing factors in Musca domestica)?

\noindent\paragraph{Response}
\textcolor{blue}{We now ...}

\noindent\subsubsection{}
Paragraph beginning L104: the sex symbols are not ideal. The combined male/female symbol has been used In other contexts to refer to "male and female" rather than representing either/or

\noindent\paragraph{Response}
\textcolor{blue}{We now ...}

\noindent\subsubsection{}
Table 2: a footnote to the table would be useful in defining the terms in the equations, some of which are far less obvious than others.

\noindent\paragraph{Response}
\textcolor{blue}{We now ...}

\noindent\subsubsection{}
L169-170. This statement makes sense, but I wonder when such an initial condition would arise, since invasion criteria are evaluated at an equilibrium state? What ancestral conditions lead to the equilibrium sex-ratio bias, upon which the invasion results are based? This is touched on later, but a statement up front will help to not leave the reader hanging.

\noindent\paragraph{Response}
\textcolor{blue}{We now ...}

\noindent\subsubsection{}
L195: define the epsilon character, which has not appeared before, or specify the size of the selection parameters relative to recombination, since this is the key point of contrast (e.g., s $<<$ r).

\noindent\paragraph{Response}
\textcolor{blue}{We now ...}

\noindent\subsubsection{}
Fig. 2, panel C. You might consider using log scale on the y-axis of the inset, since the blue and green curves do not really show.

\noindent\paragraph{Response}
\textcolor{blue}{We now ...}

\noindent\subsubsection{}
Eq. (2): the use of $V_A$ is not really useful here. Much more straightforward to place the p(1 - p) in the main equation.

\noindent\paragraph{Response}
\textcolor{blue}{We now ...}

\section{Reviewer 3}

This work (henceforth [SOO]) explores the role of both diploid and haploid selection on the evolution of new sex-chromosomes and or transitions between sex-determination systems. For that purpose, the authors develop a population genetics model of three loci with two alleles in each locus. The first locus X is the ancestral sex-determining locus, the second locus A is the locus under diploid and haploid selection. The third locus M is the novel sex-determining locus.

\noindent\subsection{}
I find the article is not clearly written and could benefit from improving the presentation of the main results. 

\noindent\paragraph{Response}
\textcolor{blue}{We now ...}

\noindent\subsection{}
I also find the article claims that: (a) the modeling work is novel and (b) the findings are unexpected to be only partially justified.

\noindent\paragraph{Response}
\textcolor{blue}{We hope the responses below help...}

\noindent\subsubsection{}
The works of Kozielska 2014 and Ubeda 2015 model the combined effects of both haploid and diploid selection on the evolution of new sex-chromosomes. The work of [SOO] is thus not novel in considering both haploid and diploid selection on the evolution of new sex-chromosomes. It does generalize previous work by considering sex-specific viability selection and get some nice analytical results.

\noindent\paragraph{Response}
\textcolor{blue}{We now ...}

\noindent\subsubsection{}
The findings in [SOO] claimed to be surprising (``Surprisingly, we find that neither force (sex ratio selection nor associations with genes that have sex-specific effects) dominates the spread of new sex-determining systems alone.") and unexpected (``Even more unexpectedly, we find that, to spread, new sex-determining alleles do not necessarily have to arise in closer linkage with genes that are differentially selected in males and females"), are rather similar to those in Ubeda et al 2015. This is a problem that repeats itself over the paper 

\noindent\paragraph{Response}
\textcolor{blue}{We now ...}

For example in:

\noindent\subsubsection{}
Page 2. The authors claim that predominant theories of the evolution of sex sex-determining systems are two: sexually-antagonistic selection and sex-ratio selection. They ignore other two: inbreeding (Charlesworth 1978) and sex-different drive (Ubeda et al 2015). 

\noindent\paragraph{Response}
\textcolor{blue}{We now ...}

\noindent\subsubsection{}
Page 3 ``Here we use mathematical models to find the conditions under which new sex-determining systems spread when individuals experience selection at both diploid and haploid stages, which allows fitness differences between the sexes and sex ratio biases to occur simultaneously"

This would be the same model as Ubeda et al 2015 adding sex-specific viability.

\noindent\paragraph{Response}
\textcolor{blue}{We now ...}

\noindent\subsubsection{}
Lines 68-69: ``If we assume that haploid selection at any particular locus predominantly occurs in one sex (e.g., meiotic drive during spermatogenesis), then such loci experience a form of sex-specific selection." 

This is a stretch and creates confusion. I consider selection acting on individuals with different sex sex-specific selection, now when selection is acting on gametes (which are neither males or females) males or females are just different environments in which selection may take place. Neither the dynamics nor the constraints are the same. 

\noindent\paragraph{Response}
\textcolor{blue}{We now ...}

\noindent\subsubsection{}
Lines 70-71: ``In this respect, we might expect that haploid selection would affect transitions between sex-determining systems in a similar manner to sex-specific diploid selection (as explored in [31,32])" 

Following up my previous comment Ubeda 2015 explores this and shows that both forms of selection are different. For example sex-antagonistic selection is needed for the invasion of a new sex-determining gene with diploid selection while antagonistic drive in each of the sexes is not needed for the invasion of a new sex-determining gene with haploid selection.

\noindent\paragraph{Response}
\textcolor{blue}{We now ...}

\noindent\subsubsection{}
Lines 77-78: ``It is not immediately clear how the spread of new sex-determining systems would be influenced by the combination of sex ratio biases and associations with haploid selected allele"

This is indeed the focus of Ubeda et al 2015.

\noindent\paragraph{Response}
\textcolor{blue}{We now ...}
\\

I will not provide further examples, but it is surprising thatI the work of Ubeda et al 2015, which is closely related to the one presented here, is cited in page 12 supporting a statement only tangential to the work of Ubeda et al 2015.

\noindent\subsection{}
In terms of modeling the distinction between meiotic drive and gametic competition is unnecessary and only complicates notation. Furthermore, from biological perspective the division seems blurry with molecular biologists considering that meiotic drive only happens in females and never in males and what is often called meiotic drive in males is gamete competition. I would suggest using a single parameter measuring transmission distortion.

\noindent\paragraph{Response}
\textcolor{blue}{We now ...}

\end{document}