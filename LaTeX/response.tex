\documentclass[10pt,letterpaper]{article}
\usepackage[top=0.85in,left=0.75in,right=0.75in,footskip=0.75in]{geometry}
\usepackage[utf8x]{inputenc}
\usepackage{amsmath,amssymb}
\bibliographystyle{plos2015}
\usepackage{color}

\begin{document}

\noindent\textbf{Color code}:\\
reviewers' words\\
%\textcolor{red}{to do/thoughts/questions between coauthors (temporary)}\\
\textcolor{blue}{authors' response}
\\

\noindent  Line numbers (L\#) in authors' comments refer to those in the revised PDF.
\textcolor{red}{check line numbers}
\\

\noindent See diff.pdf for all differences between the original submission and the revised version.

\section{Reviewer 1}

The turnover within a species from one genetic means of sex determination to another is surprisingly common, and it poses a challenging puzzle for evolutionary geneticists.  Existing theory on the topic suggests that there must be either sexually-antagonistic selection or selection variants that bring the sex ratio closer to 1:1.  This paper builds explicit models of the situation and asks about the role of selection in the haploid stage (i.e. meiotic drive and gametic selection), and finds that such effects can also serve as the primary driver of turnover of sex determining mechanisms.  While this is a rather focused topic, it is a striking evolutionary puzzle, and the paper presents a very satisfying solution that is much more general than we had previously thought.  It also overturns the primacy of Fisherian sex ratio evolution as the driver of the dynamics of genes of this sort.

The authors devise a satisfyingly simple genetic scenario, with one ancestral sex-determining locus, one locus under selection, and a third locus for the nascent sex determining locus.  They manage to collapse some messy algebra into terms like the ``haplotypic growth rate" that allow the reader to intuit many of the results.  I only have a few suggestions:

\textcolor{blue}{
We thank the reviewer for their careful consideration of our manuscript and pithy summary. We have incorporated the sentiment of ``a striking evolutionary puzzle'' to help frame our study, L51. Below, we itemize further improvements we have made that correspond to more specific reviewer comments. 
}

\noindent\subsection{}
The selected locus allows for three modes of selection: gametic selection (weighting the frequencies of gamete genotypes), the usual diploid viability selection, and meiotic drive (where heterozygotes may produce gametes that deviate from 50:50 proportions).  It should be noted that the model does not accommodate fertility or sexual selection.

\noindent\paragraph{Response}
\textcolor{blue}{
One could interpret our diploid selection terms as a combination of viability and fertility selection, where fertility selection is independent of the mating partner.
However, fertility selection that depends on the mating pair would require a different formulation.
We now note this in the model section (L120-122):
``Diploid males and females then experience viability and/or individual-based fertility selection, with relative fitnesses $w_{AA}^{\circ}$, $w_{Aa}^{\circ}$, and $w_{aa}^{\circ}$.
We do not consider fertility selection that depends on the mating partner, e.g., sexual selection with variation in choosiness."
}

\noindent\subsection{}
The ``haplotypic growth rate" plays a really essential role in providing intuition about the results. Similarly the $\chi_{mi}^{(XY)}$ terms also are crucial.  I suggest giving a short name for the chi terms (currently they are the ``rate at which mutant haplotypes on background i recombine onto the other A locus background in heterozygotes"...).   The importance of these becomes clear in Table 2, but I wonder if there is a graphical way to make their meaning even clearer.  

\noindent\paragraph{Response}
\textcolor{blue}{
We now refer to the $\chi_{mi}^{(XY)}$ terms as the ``dissociative force" as they break down linkage disequilibrium (L161). 
We have attempted various graphical schemes for representing the three loci in two sexes and the associations that evolve between them in various scenarios. 
However, apart from Fig 1, we found these too complicated or opaque to be useful. 
Instead, we have substantially restructured the results section and reduced the technicality (see responses XXXXXX), which should generally help to make the important messages clearer. 
Specific to this point, in the paragraph 179-199, we itemize three take-home messages from table 2 about the ``haplotypic growth rates'' and ``dissociative forces''. 
}

\noindent\subsection{}
It needs to be stated somewhere that the models are entirely deterministic, implying that the authors assume that the selective forces always exceed the force of random drift. It might be useful to discuss this a bit. Is there evidence that in no case has drift been a key factor in changes of sex determination system?  The discussion mentions alleles of weaker effect, but does not mention that in this scenario, models of weak effects would need to consider drift as well.

\noindent\paragraph{Response}
\textcolor{blue}{
This is true, we now mention that the model is deterministic, and hence ignores drift (L130-131). 
As suggested, when discussing the model limitations at the end of the Discussion we now mention a study on the effect of drift during trans-GSD transitions, which tend to drift in the direction of epistatically-dominant sex-determining loci (L538-541): ``It should be noted, however, that the dynamics of sex-determining alleles with very weak effect will be influenced by genetic drift, which itself has been shown to cause trans-GSD transitions (Veller et al 2017)."}

\noindent\subsection{}
A glitch appears in the section ``Loose linkage with the ancestral sex-determining region" on line 270, where the text reads, ``To leading order in selection, the leading eigenvalues are..."

\noindent\paragraph{Response}
\textcolor{blue}{Thank you. We have dropped the words ``the leading eigenvalues are".}

\noindent\subsection{}
Gametic selection among sperm is a bit different from sperm competition, since the latter involves pair-wise (or multi-way) bouts, whereas gametic selection assigns constant fitnesses to each haplotype. I would replace occurrences of ``sperm competition" with ``male gametic selection", or maybe ``sperm gametic competition". To be consistent, replace ``egg competition" with ``female/egg gametic competition". 

\noindent\paragraph{Response}
\textcolor{blue}{We have made the suggested replacements.
In addition, we have added a more detailed discussion of how gametic competition relates to the number of mating partners involved, L108-118 (see also response XXXXXX). 
Specifically, we note that our model explicitly considers within-ejaculate competition or competition among sperm from many males but not competition among sperm from pairs (or small numbers) of males. 
%n the introduction we mention that sperm competition has been shown to alter allele frequencies and fitness, and in this case we now mention this was shown within a single ejaculate (L68).
}

\noindent\subsection{}
I am not certain of this, but it seems that a potentially important difference between ESD and GSD has to do with the variance in proportion male? Since ESD (in a constant environment, as is implicitly modeled here) has a binomial sampling of sexes, but GSD might deviate from this.  This might be an issue because Fisherian sex ratio selection acts on the mean and not the variance.

\noindent\paragraph{Response}
\textcolor{blue}{In our deterministic model (where the number of binomial trials is infinite), the variance in the fraction male is 0 with ESD, as it is with GSD. 
We now mention this (L373-375): ``In our deterministic model this means the fraction female in the subpopulation containing $m$ is exactly $k$, even when $m$ is rare (i.e., ESD does not introduce any additional variance in sex determination)." 
 }

\noindent\subsection{}
Full sharing of the Mathematica file is to be commended.

\section{Reviewer 2}

Scott et al. present a 3-locus model for sex-determination (SD) evolution (1 locus has an ancestral SD function; 1 has a derived SD function; 1 locus affects selection in the haploid or diploid phase of the life cycle). Their results converge on previously-identified conditions for the spread of new SD alleles, while also broadening the range of scenarios and conditions that can favour transitions between SD systems.

The paper is interesting and conceptually appropriate for PLoS Biology. The authors have done a great job of analysing a somewhat complicated model and drawing out a relatively complete picture of the conditions favouring the evolution of new sex determination systems, and I expect their expert treatment will go over well with people that are close to the research context of the new theory.

On the other hand, I am much less sure that this paper will reach beyond experts in this subject area, though the authors should seek to broaden the appeal. This is not because the results are narrow -- I think they can have a broad appeal. Rather, the paper is written in a technical manner that is likely to limit the appeal beyond the expert class. I am reasonably comfortable with models like this, but still found the presentation to be very hard going in a variety of places, particularly in the results. I offer some suggestions for presentation that may help improve the presentation and broaden the potential appeal of the paper.

\textcolor{blue}{
We thank the reviewer for their thoughtful, positive, and very constructive comments. We have substantially restructured the Introduction, Model, and Results in response to these suggestions (and comments from other reviewers), as outlined below. 
This has significantly improved the accessibility and appeal of our results. 
}

\noindent\subsection{}
I think the introduction could be simplified so that the context of the study is more obvious. The authors essentially need to make three key points, which are there, but could be better drawn out. First, transitions between sex-determining systems are common, and recent empirical work has given us a much greater appreciation for just how common these transitions are. Second, there are two major mechanisms that are thought induce evolution of new sex-determination mechanisms: (1) selection to stabilize the sex ratio, and (2) hitchhiking between (tightly linked) sex-determination and sexually antagonistic alleles. Third, the authors should clearly outline the major limitations of current theory as a means of motivating their mega-model that captures an array of genetic and selection scenarios that may lead to transitions in sex determination. As I mentioned, these elements can all be found in the introduction, but the pitch can be substantially streamlined to allow readers to easily see: (1) what is the context? (2) What do we currently think drives transitions between SD systems? And (3) what are the holes in the theory and how will the authors address them?

\noindent\paragraph{Response}
\textcolor{blue}{
We have simplified the introduction according to the suggested format. In addition, when discussing holes in current theory we have expanded our discussion of previous studies that include haploid selection, as suggested by reviewer 3. 
To accommodate these changes, we have moved the discussion of empirical evidence for haploid selection to the discussion (L441-454). The introduction is now structured as follows:
\begin{enumerate}
\item[1.]{The first paragraph discusses how common transitions are, and now mentions that recent empirical evidence is making this increasingly apparent. (L2-20)}
\item[2-3.]{The second and third paragraphs lay out the two major mechanisms thought to drive transitions in sex-determination, (1) selection to stabilize the sex ratio, and (2) linkage associations between alleles favoured in one sex and the alleles that determine that sex. (L21-29, L30-38) }
\item[4.]{Connecting to point (2), we then found it useful to introduce one novel aspect of our model -- we consider a selected locus that is very tightly linked to the ancestral sex-determining locus and find the surprising result that transitions in which linkage is reduced can be favoured. (L39-48)}
\item[5.]{Next, we introduce the most significant novel model feature, haploid selection, which we present as an evolutionary problem in light of (1) and (2). (L49-56) }
\item[6.]{The sixth paragraph then describes previous studies that have included haploid selection (see also reviewer 3 recommendations). These studies suggest that (1) and (2) are both very important with haploid selection and point to the major limitations in solving the evolutionary puzzle. (L57-69)}
\item[7.]{Finally, we lay out our main results in this context. This makes it clear that our general model is necessary to determine the important forces driving transitions in sex determination (e.g., considering various genomic locations). We particularly highlight the unexpected conclusions that our model allows us to reach. (L70-83)}
\end{enumerate}
}

\noindent\subsection{}
Likewise, the results yield three major insights:\\
(1) SD transitions can, counterintuitively, potentially INCREASE sex ratio bias\\
(2) Tight linkage and sexual antagonism are not required for transitions to occur\\
(3) Under haploid selection, conditions for the spread of new sex-determining alleles are extremely permissive.\\
Again, all of these points are there, but they could be better drawn out than they currently are. The results, as written, are hard-going, and I expect will be particularly challenging for readers that are not heavily immersed in evolutionary theory.

\noindent\paragraph{Response}
\textcolor{blue}{
As discussed further below (response XXXXX), we have highlighted conclusions in the results section as suggested. 
To further reduce the technicality of our presentation, we have also revised the results section. 
In particular, we have restructured the presentation of the subsections ``Generic invasion by a neo-Y or neo-W'' and ``Tight linkage with the ancestral sex-determining locus''. 
In the ``Generic invasion by a neo-Y or neo-W" section, we have moved some details to the supplementary material (particularly, proofs regarding the leading eigenvalue, formally LXXX-XXX). 
Now, we focus on explaining the intuition behind our technical results.  
For example, we first discuss the `haplotypic growth rates' and 'dissociative forces' (L158-170 and `interpreting this condition...', L173).
We then lay out four specific conclusions (L179-199). 
In the ``Tight linkage'' section, we now first describe specific cases why neo-W's can have improved female fitness, Conclusion 1, L235-259, with more focus on non-technical interpretation (e.g., L237-240). 
Then we discuss the fact that looser sex-linkage can evolve (L260-275, Conclusion 2). 
This re-organisation allows us to break down the logic behind Conclusion 2 more clearly. 
}

\noindent\subsection{}
One potential way to help readers -- particularly those interested in what the paper has to say, but are not mathematically inclined -- is to conspicuously highlight each important result and briefly place it into context of earlier theory. But at the same time, the authors have done a great job of drawing out general mathematical results from a relatively complicated model -- their mathematical insights ARE the results and I am not suggesting that they be minimized or shunted into appendices.

One tactic for including analytical details which specialists will want to see, while allowing less mathematically inclined readers to easily find each main result, is to place short and accessible summaries at the close of each section of the Results. Such a tactic has been put to good use in a hugely influential paper by Allen Orr (1998 Evolution), which I expect the authors will be familiar with. Orr presents the major mathematical results of his analysis alongside prominently labelled punchline paragraphs that summarize each of his main results in a concise and easily understandable manner. If the authors choose to take this approach, please be clear to place the take-home messages in context: clearly state whether the result confirms previous theory, or represents a new result or contradiction of an earlier result.

\noindent\paragraph{Response}
\textcolor{blue}{
This is an excellent suggestion and has greatly improved the clarity of presentation in our manuscript. In the results, we have specifically highlighted our main conclusions in the style of Orr 1998. We have chosen slightly different conclusions from the three suggested by the reviewer: 
%We now give 3 main conclusions at the end of the introduction (L69) which echo those mentioned by the reviewer (looser linkage can evolve, equal selection to increase or decrease the sex ratio, ESD does not always invade biased sex ratio; all of which suggest haploid selection makes conditions for transitions more permissive).
\begin{enumerate}
\item[(1)] Selection on loci in or near the non-recombining region around the ancestral sex-determining locus ($r\approx0$) prevents cis-GSD transitions (XY $\leftrightarrow$ XY, ZW $\leftrightarrow$ ZW) but can spur trans-GSD transitions (XY $\leftrightarrow$ ZW). (L236-238)
\item[(2)] With tight linkage between a selected locus and the ancestral sex-determining locus ($r\approx0$), trans-GSD transitions (XY $\leftrightarrow$ ZW) can be favoured by selection \textit{even if they weaken sex-linkage} ($r<R$), potentially shifting sex determination to a different chromosome ($R=1/2$). 
Such transitions can also lead to the maintenance of multifactorial sex-determination systems. (L290-294)
\item[(3A)] New sex-determining alleles (causing cis-GSD transitions, XY $\leftrightarrow$ XY or ZW $\leftrightarrow$ ZW) are favoured if they arise more closely linked with a locus that experiences (haploid and/or diploid) selection than the ancestral-sex-determining locus is ($R<r$). (L326-328)
\item[(3B)] New sex-determining alleles (causing trans-GSD transitions, XY $\leftrightarrow$ ZW) are favoured if they are linked with an ancestrally-autosomal locus that experiences (haploid and/or diploid) selection ($R<1/2$, $r=1/2$). (L333-335)
\item[(3C)] With haploid selection, new sex-determining alleles (causing trans-GSD transitions, XY $\leftrightarrow$ ZW) can spread even if they arise less closely linked with a locus that experiences selection than the ancestral-sex-determining locus is ($r<R$). (L341-343)
\item[(4)] When selection is weak relative to recombination, the presence of haploid selection equally favors the spread of new sex determination systems that reduce sex-ratio bias (benefiting from Fisherian sex ratio selection) or generate a sex-ratio bias (benefiting from associations with selected alleles). (L350-353)
\item[(5)] Transitions from genetic to environmental sex-determination are not straightforwardly predicted by selection to balance the zygotic sex ratio when haploid selection is present.  (L411-413)
%\vspace{0.25cm}\\
\end{enumerate}
These conclusions are also more clearly laid out in the introduction (L48-49, L74-85) and are referenced directly in summarisations in the discussion (L424-432, L543-554). 
}

\textcolor{blue}{Of these conclusions, (3A), (3B), and (3C) have some precendent. (3A) and (3B) are similar to studies by van Doorn and Kirkpatrick (2007,2010) that do not include haploid selection. These comparisons are made immediately above the conclusions (L316-330) and mentioned elsewhere (L55, L77, L489). 
Including haploid selection, Ubeda et al. (2015) also involved examples similar to (3A) and (3B), except that the starting state was ESD. 
Kozielska et al. (2010) also had numerical examples that involved trans-GSD transitions with ($R>r$), as stated in (3C), but focussed only on sex ratio selection. 
We discuss how our results compare to these studies in the introduction (L58-75), discussion (L433-447) and results (L193, L363-369).
Nevertheless, we have highlighted these conclusions, partly for clarity of communication and partly because our results are more general. 
That is, as stated by the reviewer, these results ``converge on previously-identified conditions for the spread of new SD alleles, while also broadening the range of scenarios and conditions that can favour transitions between SD systems''}
%As (1A), (1B), and (2) have some antecedents in the literature, we discuss these immediately below each conclusion. 
%We also discuss, immediately below each conclusion, its novelty.
%These are all new results, although (1A) was shown for the special case of $r=0$, $R=1/2$, and ploidally-antagonistic selection in Kozielska et al 2010, which we now mention immediately below (1A).
}


\subsection{MINOR COMMENTS}

\noindent\subsubsection{}
L3: "of diploid sexes" can be deleted

\noindent\paragraph{Response}
\textcolor{blue}{Deleted.}

\noindent\subsubsection{}
L12: missing a ``)"

\noindent\paragraph{Response}
\textcolor{blue}{Fixed.}

\noindent\subsubsection{}
L35-37: sentence could use an edit for ease of reading.

\noindent\paragraph{Response}
\textcolor{blue}{We have replaced ``Thus, if the population sex ratio is biased towards one sex, the average per-individual contribution of genetic material to the next generation from the opposite sex is greater" with ``Thus, if the sex ratio is biased, an individual of the rarer sex will, on average, contribute more genetic material to the next generation". (L26-28)}

\noindent\subsubsection{}
Immediately preceding the paragraph beginning line 44: It would be worth establishing the limitations of previous work. What exactly motivated you to do this mathematical modelling?

\noindent\paragraph{Response}
\textcolor{blue}{We have rearranged and rewritten the introduction to better motivate our model (see Response 2.1). In particular, paragraph 5 presents the role of haploid selection as an evolutionary puzzle and paragraph 6 explains the limitations of previous attempts to solve the puzzle. (L49-69)}

\noindent\subsubsection{}
L56-57: Rephrase for clarity "individual cases of meiotic drive are generally sex-limited, and exclusively effect male or female gametes". And on line 57, "sex specific" could be changed to "sex-limited".

\noindent\paragraph{Response}
\textcolor{blue}{
Thank you for this suggestion, ``sex-limited'' is a more efficient way to communicate that haploid selection often only occurs in gametes produced by one sex. 
We now use ``sex-limited'' throughout, with two exceptions. L72 and L126, where we are referring to the fact that we model haploid selection in either/both sexes (focussing on sex-limited cases in our examples). 
This section was re-written but the section that roughly corresponds to the highlighted sentence now reads: ``...haploid selection is typically sex-limited in that it usually occurs among gametes produced by one sex only." (L51-52) 
}

\noindent\subsubsection{}
Lines 80-81: This primary finding is completely aligned with the earlier views of theory that are summarized above. Are you sure you want to lead with this as the main emphasis of the study? It understates what is new.

\noindent\paragraph{Response}
\textcolor{blue}{
As suggested above, we have re-written to focus on our most novel findings, which are now laid our more specifically, e.g., L71-85 (see also Responses 2.1 and 2.2). 
}

\noindent\subsubsection{}
M locus is dominant to X. How important is this assumption? Is this compatible with any data (e.g., on feminizing or masculinizing factors in Musca domestica)?

\noindent\paragraph{Response}
\textcolor{blue}{
The `epistatic-dominance' between loci is important to the evolutionary dynamics. For example, Veller et al. (2018, Genetics) show that there is a tendency for drift to occur in the direction of the epistatically-dominant sex-determining locus (mentioned on L545-546). Our results also show that the outcome can depend on which sex-determining locus has an epistatically-dominant effect. For example, the end state of the red line in Fig 5 is the starting condition of the other panel. That is, we find that recurrent invasion can occur by a sequence of sex-determining loci that are epistatically-dominant over their predecessors (requiring haploid selection). 
 }
 
\textcolor{blue}{
Empirical systems with multi-factoral sex-determination (including \textit{Musca domestica}) do indeed indicate that derived sex-determining systems are often `epistatically-dominant'. 
We now mention this where we introduce this aspect of the model (L98-109). 
We also point to the supplementary Mathematica file, which allows arbitrary `epistatic dominance' to be specified. 
There are some numerical examples with reduced `epistatic dominance' in the supplementary material of van Doorn \& Kirkpatrick (2007, Nature), which we also point to here.
Their findings suggest that the selective advantage is reduced (the modifier mutation effectively has a smaller effect) but the direction of evolution is expected to be qualitatively unchanged. 
}

\noindent\subsubsection{}
Paragraph beginning L104: the sex symbols are not ideal. The combined male/female symbol has been used In other contexts to refer to "male and female" rather than representing either/or

\noindent\paragraph{Response}
\textcolor{blue}{This is true, the male/female symbol is commonly used to represent hermaphrodites, we have replaced the male/female symbol with a circle, which is a basis for both the male and female symbols.}

\noindent\subsubsection{}
Table 2: a footnote to the table would be useful in defining the terms in the equations, some of which are far less obvious than others.

\noindent\paragraph{Response}
\textcolor{blue}{We have extended the footnote to define all terms in Table 2.}

\noindent\subsubsection{}
L169-170. This statement makes sense, but I wonder when such an initial condition would arise, since invasion criteria are evaluated at an equilibrium state? What ancestral conditions lead to the equilibrium sex-ratio bias, upon which the invasion results are based? This is touched on later, but a statement up front will help to not leave the reader hanging.

\noindent\paragraph{Response}
\textcolor{blue}{We have flipped the statement to talk about male biased sex-ratios instead, and mention that this is possible at equilibrium in an XY system under the more familiar scenario of meiotic drive in males. (L183-187)}

\noindent\subsubsection{}
L195: define the epsilon character, which has not appeared before, or specify the size of the selection parameters relative to recombination, since this is the key point of contrast (e.g., s $<<$ r).

\noindent\paragraph{Response}
\textcolor{blue}{We have replaced this instance of $\epsilon$ with the more familiar $s<<r$ style assumption (L210). When we introduce $\epsilon$ in the weak selection section we now verbally explain that it is just a number that is much less than one (L295-296).}

\noindent\subsubsection{}
Fig. 2, panel C. You might consider using log scale on the y-axis of the inset, since the blue and green curves do not really show.

\noindent\paragraph{Response}
\textcolor{blue}{The blue and green curves should not really show as they are not expected to increase from a very small initial frequency. We have kept the original scale but now state in the legend that the blue and green loci do not invade.}

\noindent\subsubsection{}
Eq. (2): the use of $V_A$ is not really useful here. Much more straightforward to place the p(1 - p) in the main equation.

\noindent\paragraph{Response}
\textcolor{blue}{We have made the suggested change.}

\section{Reviewer 3}

This work (henceforth [SOO]) explores the role of both diploid and haploid selection on the evolution of new sex-chromosomes and or transitions between sex-determination systems. For that purpose, the authors develop a population genetics model of three loci with two alleles in each locus. The first locus X is the ancestral sex-determining locus, the second locus A is the locus under diploid and haploid selection. The third locus M is the novel sex-determining locus.

\noindent\subsection{}
I find the article is not clearly written and could benefit from improving the presentation of the main results. 
I also find the article claims that: (a) the modeling work is novel and (b) the findings are unexpected to be only partially justified.
The works of Kozielska 2014 and Ubeda 2015 model the combined effects of both haploid and diploid selection on the evolution of new sex-chromosomes. The work of [SOO] is thus not novel in considering both haploid and diploid selection on the evolution of new sex-chromosomes. It does generalize previous work by considering sex-specific viability selection and get some nice analytical results.

\noindent\paragraph{Response}
%\textcolor{red}{The abstract and summary still seem to be a little over-selling...}
\textcolor{blue}{
We have performed an extensive re-arrangement to improve the clarity of our writing, especially regarding the main results (see Responses 2.1-2.3). 
As suggested by the reviewer (here and below), we also now give more prominence to the related results by Kozielska et al 2010 and Ubeda et al 2015 (e.g., L56-71, L198, L367-373, L440-446). 
In particular, we introduce these studies on Lines 56-71 and more precisely indicate which of our model features and results are new and unexpected in light of these previous results.
}

\textcolor{blue}{
While related, these two studies considered particular cases that we do not consider at length. 
For example, Ubeda et al. (2015) consider ESD-GSD transitions. 
Nevertheless, the transitions in Ubeda et al. (2015) involve increased linkage with a haploid selected locus, which is precedent for our Conclusions 3A and 3B (which are for GSD-GSD transitions in which linkage with a haploid selected locus is increased).
We now mention this in the Introduction (L59-61), Results (L369-373), and Discussion (L432).  
Similarly, Kozielska et al. (2010) involves a special case of GSD transitions during which sex-linkage goes from perfect to none, which is is an example of our Conclusion 2 (and similar to Conclusion 3C, which are conclusions regarding GSD-GSD transitions in which linkage with a haploid selected locus is decreased).
This is mentioned on lines (L63-64, L303, and L368). 
We believe that our new presentation makes the novel and unexpected features of our results much more apparent. 
For example, despite these precedents, we find the combination of Conclusions 3A, 3B, and 3C to be a striking solution that reconciles the results of Kozielska et al. (2010) and Ubeda et al. (2015). 
We elaborate further on what makes our results novel and surprising in the responses below.
}

\noindent\subsubsection{}
The findings in [SOO] claimed to be surprising (``Surprisingly, we find that neither force (sex ratio selection nor associations with genes that have sex-specific effects) dominates the spread of new sex-determining systems alone.") ... [but this result is rather similar to those in Ubeda et al 2015]. 

\noindent\paragraph{Response}
\textcolor{blue}{
Here we were referring to the result that $\lambda_{W'}^{XY} = \lambda_{Y'}^{ZW}$ with haploid selection, even when sex-ratio biases evolve to become either stronger or weaker (e.g., $\lambda_{Y'}^{ZW}$ vs. $\lambda_{W'}^{XY}$ with $R<r$ and male meiotic drive) or when sex-ratio biases are either present or absent (e.g., $\lambda_{W'}^{XY}$ vs. $\lambda_{Y'}^{ZW}$ with $r<1/2$, $R=1/2$, and male meiotic drive).  
As mentioned by the reviewer, Step 2 in Ubeda et al 2015 shows that associations with a haploid selected locus can favour a new sex-determining allele despite causing biased sex ratios. 
Nevertheless, it was difficult for us to intuit whether the presence/absence of ancestral sex-ratio bias would have a negligible or overwhelming effect on the evolution of new sex-determining alleles. 
Because we start from GSD, our model allows ancestral sex-ratio biases to build up due to the presence of haploid selected loci near to the ancestral sex-determining locus. 
There are reasons to suspect that ancestral-sex-ratio biases would strongly influence the spread of new sex-determining systems.
For example, step 3 of Ubeda et al involves the spread of an autosomal meiotic drive suppressor favoured by Fisherian sex-ratio selection and Kozielska et al 2010 show that ancestral sex-ratio bias can favour new sex determining systems that balance the sex ratio.
We found our result, that these forces are exactly equal (with selection weak relative to recombination), such that there is no difference between the propensity for XY-ZW or ZW-XY transitions, to be surprising.
To clarify our meaning we have rephrased (in Author summary): ``Surprisingly, we find the two forces (selection to equalize the sex-ratio and the benefits of hitch-hiking with driven alleles that distort the sex ratio) will often be equally strong, and thus neither is sufficient to explain the spread of new sex-determining systems in every case." 
We believe that the re-framing of the Introduction/Discussion and the emphasis given to specific Conclusions will help to clarify this point throughout the paper.
}

\noindent\subsubsection{}
\noindent and unexpected (``Even more unexpectedly, we find that, to spread, new sex-determining alleles do not necessarily have to arise in closer linkage with genes that are differentially selected in males and females"), are rather similar to those in Ubeda et al 2015. This is a problem that repeats itself over the paper 

\noindent\paragraph{Response}
\textcolor{blue}{
The Ubeda et al. (2015) results do indeed show that differential selection between male and female diploids is not required for new sex-determining alleles to spread; they show that a new genetic sex-determining allele can spread from ancestral ESD when it is linked to a locus that experiences sex-specific meiotic drive. 
In this sentence, we were attempting to stress that new sex determining alleles can spread when they are \textit{less closely linked or even completely unlinked} to loci under differential male/female haploid and/or diploid selection. 
To our knowledge, this has only been shown for the special case of male meiotic drive in an ancestral XY system with $r=0$ and $R=1/2$ with a neo-W, which is driven by sex-ratio selection (Kozielska et al 2010). 
We therefore found the large parameter space allowing invasion with $r<R$, sometimes in the absence of sex-ratio selection (or even in the absence of haploid selection), to be very unexpected.
To improve our communication of this result, we have rephrased (in Author Summary): ``We also find that new sex-determining alleles can spread despite being less closely linked to selected loci, driven by initially tight linkage or haploid selection." 
The changes to the presentation of the Introduction, Results, and Discussion should also help to clarify this point further (see Responses 2.1 and 2.2), this result is now also highlighted as `Conclusion 4'. 
}

\noindent\subsubsection{}
For example in: Page 2. The authors claim that predominant theories of the evolution of sex-determining systems are two: sexually-antagonistic selection and sex-ratio selection. They ignore other two: inbreeding (Charlesworth 1978) and sex-different drive (Ubeda et al 2015). 

\noindent\paragraph{Response}
\textcolor{blue}{
Here, sexually-antagonistic selection was referred to as an example of ``selective differences between the sexes". 
We believe the phrase 'selective differences between the sexes' includes sex-differences in drive (Ubeda et al 2015), which occurs during haploid rather than diploid selection, as indicated in Fig 1B (see also Response 3.1.6).
As mentioned elsewhere, the enhanced discussion of Ubeda et al. (2015) should help to clarify this point. 
Charlesworth and Charlesworth 1978 (Am Nat and Heredity) discuss the role of inbreeding (and particularly inbreeding depression) in transitions from hermaphroditism to dioecy. 
As described in the model section and lines 468-472 of the original submission, we have only considered GSD-GSD transitions and GSD-ESD transitions.
For clarity, we have moved this description into the Introduction (L49-50). 
Where we draw a parallel between ESD and hermaphroditism (L468-476), we suggest future theory include inbreeding and cite Charlesworth and Charlesworth (1978). 
}

\noindent\subsubsection{}
Page 3 ``Here we use mathematical models to find the conditions under which new sex-determining systems spread when individuals experience selection at both diploid and haploid stages, which allows fitness differences between the sexes and sex ratio biases to occur simultaneously".
This would be the same model as Ubeda et al 2015 adding sex-specific viability.

\noindent\paragraph{Response}
\textcolor{blue}{
As we now make clear, we do not look at the same situation as Ubeda et al 2015 and our contribution is not just to add sex-specific viability. 
In particular, Ubeda et al start from ESD. 
Therefore, associations with an ancestral sex-determining locus cannot build up. 
These lead to some of our most surprising results (e.g., haploid selection favours trans-GSD transitions for a wider variety of genomic locations than diploid selection alone; ESD does not invade despite initial sex-ratio bias).
Further, our model and intuition-giving analytical solutions explore a much wider range of scenarios than the numerical results of Ubeda et al, allowing, for example, sex-specific selection (e.g., sex-antagonism) and various genomic arrangements (any order/linkage of the three loci).
Our model is thus much more general and clarifies just when one should expect to see the numerical results of Ubeda et al.
We have rephrased to make this clear (after discussing Ubeda et al 2015 in the Introduction, see Response 2.1; L70-74): ``Here, we analytically find the conditions under which new GSD or ESD systems spread in ancestral GSD systems with generic linkage between the loci involved and generic sex-specific haploid and diploid selection. 
Doing so, we reconcile the results of Kozielska et al.\ (2010) and Ubeda et al.\ (2015) by showing when new GSD systems that increase \textit{or} decrease linkage with loci under haploid selection spread."
}

\noindent\subsubsection{}
Lines 68-69: ``If we assume that haploid selection at any particular locus predominantly occurs in one sex (e.g., meiotic drive during spermatogenesis), then such loci experience a form of sex-specific selection." 

This is a stretch and creates confusion. I consider selection acting on individuals with different sex-specific selection, now when selection is acting on gametes (which are neither males or females) males or females are just different environments in which selection may take place. Neither the dynamics nor the constraints are the same. 

\noindent\paragraph{Response}
\textcolor{blue}{In this study, we are referring to organisms, such as animals and angiosperms, in which the type of gamete depends on the sex of the diploid that produced it. In this respect, one could consider gametic selection among sperm, for example, to be a form of male-specific selection. 
However, as noted by the reviewer here, the dynamics of selection during gametic selection cannot be captured by a diploid selection coefficient, which is why we specifically model selection upon haploid genotypes.
In our new manuscript, a roughly corresponding statement now reads ``On one hand, haploid selection is typically sex-limited in that it usually occurs among gametes produced by one sex only [38-41].''.
We have also expanded our explanation of the mating system and the way we model haploid selection (see Response 3.2 below).
}

\noindent\subsubsection{}
Lines 70-71: ``In this respect, we might expect that haploid selection would affect transitions between sex-determining systems in a similar manner to sex-specific diploid selection (as explored in [31,32])" 

Following up my previous comment Ubeda 2015 explores this and shows that both forms of selection are different. For example sex-antagonistic selection is needed for the invasion of a new sex-determining gene with diploid selection while antagonistic drive in each of the sexes is not needed for the invasion of a new sex-determining gene with haploid selection.

\noindent\paragraph{Response}
\textcolor{blue}{In the absence of sex-differences in selection it is true that antagonistic drive is not needed, but \textit{differences} in haploid selection between the sexes are needed (e.g., drive in only one sex, with variations maintained by ploidally-antagonistic selection). 
If drive was the same in both sexes, $\alpha^{m}=\alpha^{f}$ (and all other male/female selection terms also equal), we show that the new sex-determining allele would not invade (e.g., equation 2 and 3). 
In this sense haploid selection is similar to sex-specific diploid selection, and together haploid and diploid selection determine the differences in male vs.\ female fitness, which can drive sex-determination transitions (see Response 3.2.4).
This sentence meant to introduce the `sex-specific' aspect of haploid selection in the way Ubeda et al 2015 considers it; associations with alleles favoured in one sex (or the gametes produced by that sex) should favour a new sex-determining allele.
We have rephrased to make this clear (L52-54): ``Therefore, one might expect new sex-determining systems to benefit from close linkage with haploid selected loci, as found for sex-differences in diploid selection."
We now follow this sentence up with a paragraph that clarifies its meaning by comparing with the results of Kozielska et al 2010 and Ubeda et al 2015 (see Response 2.1).}

\noindent\subsubsection{}
Lines 77-78: ``It is not immediately clear how the spread of new sex-determining systems would be influenced by the combination of sex ratio biases and associations with haploid selected allele"

This is indeed the focus of Ubeda et al 2015.

\noindent\paragraph{Response}
\textcolor{blue}{While Ubeda et al 2015 did focus on sex ratio biases and haploid selected alleles, the system started with ESD, where there are no associations (see Response 3.1.4).
Further, the restricted scope of that numerical study prevents us from generalizing beyond its special case (see Response 3.1.4).
We now present a refurbished argument for why it is difficult to intuit the implications of haploid selection while summarizing the results of Ubeda et al and Kozielska et al 2010 in the Introduction (see Response 2.1).
}

\noindent\subsubsection{}
I will not provide further examples, but it is surprising thatI the work of Ubeda et al 2015, which is closely related to the one presented here, is cited in page 12 supporting a statement only tangential to the work of Ubeda et al 2015.

\noindent\paragraph{Response}
\textcolor{blue}{Ubeda et al 2015 was already cited on line 396 of the original submission, where we mentioned that it showed that sex-ratio biases could become more severe during a transition.
Nevertheless, we now discuss Ubeda et al 2015 in the Introduction, Results, and Discussion (see Responses above).}

\noindent\subsection{}
In terms of modeling the distinction between meiotic drive and gametic competition is unnecessary and only complicates notation. Furthermore, from biological perspective the division seems blurry with molecular biologists considering that meiotic drive only happens in females and never in males and what is often called meiotic drive in males is gamete competition. I would suggest using a single parameter measuring transmission distortion.

\noindent\paragraph{Response}
\textcolor{blue}{
We have chosen not to adjust our models to include a single parameter that measures transmission distortion. 
Given our definition of meiotic drive and gametic competition (L106-116), the two processes are different.
In particular, meiotic drive can only have an effect in heterozygotes.
For example, this means that cases with `sperm killers' and then ``gametic competition" amongst the gametes produced by a single individual are appropriately modelled by our `meiotic drive' parameter. 
For the same reason, ``gametic competition'' of any type in a monogamous mating system corresponds to `meiotic drive' in our model (L116-119). 
On the other hand, in our model, ``gametic competition'' is competition amongst all the gametes produced by one sex (e.g., pollen from many male plants).
Thus, for example, allele frequency changes can occur by gametic competition even if all individuals were homozygous (as long as there is a polymorphism).
We have extended our description of how we use the terms `gametic competition' and `meiotic drive' and how they are modelled in the model section (L106-116), which helps to clarify why they are different and under what circumstances each applies. 
Unfortunately, beyond defining our terms, it was not straightforward to resolve the different usages of the terms by different communities.
}





\end{document}