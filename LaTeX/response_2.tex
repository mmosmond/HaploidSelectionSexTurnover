\documentclass[10pt,letterpaper]{article}
\usepackage[top=0.85in,left=0.75in,right=0.75in,footskip=0.75in]{geometry}
\usepackage[utf8x]{inputenc}
\usepackage{amsmath,amssymb}
\usepackage[dvipsnames]{xcolor}

\title{Response to the Editor}
\date{}

\date{\vspace{-5ex}}

\begin{document}

\maketitle

\noindent
Here we provide a response to the editors suggestion. The academic editors comment has been provided in full (black text) with author response in (\textcolor{blue}{blue text}).  Line numbers (L\#) refer to those in the revised PDF.

As recommended, we also provide a file highlighting the difference between the original submission and the revised version (diff.pdf), which was generated using LaTeXdiff.

\section{Academic editor's comment}

\noindent\subsection{}

I like the style of condensing model results to single-sentence ``conclusions", but here one also has to check if the condensed format works (given that the tendency is that many people skim through a paper instead of digesting it fully). In Conclusion 4, the meaning of ``equally favors" is quite ambiguous and consequently hard to understand. It might be better phrased as a statement about whether the sex ratio bias caused by the new sex determination system has an impact on the likelihood it will spread, rather than something favouring something that causes something. Same thing, but easier to interpret?

\noindent\paragraph{Response}
\textcolor{blue}{
We thank the editor for pointing this out and have revised to make the meaning clearer for those readers who skim the surrounding text.
In particular, we have replaced ``haploid selection equally favors the spread of new sex determination systems that reduce sex ratio bias (benefiting from Fisherian sex ratio selection) or that generate a sex-ratio bias (benefiting from associations with selected alleles" with ``trans-GSD transitions in the presence of haploid selection are favoured as often and as strongly whether they erase ancestral sex-ratio bias (benefiting from Fisherian sex ratio selection) or generate sex-ratio bias (benefiting from associations with selected alleles)" (L366-369).
We have chosen to use the phrase ``favoured as often and as strongly" instead of something involving ``likelihood" because this is a deterministic model.
}

\end{document}