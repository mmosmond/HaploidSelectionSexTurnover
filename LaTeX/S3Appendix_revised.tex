\documentclass[10pt,letterpaper]{article}
\usepackage[top=0.85in,left=2.75in,footskip=0.75in]{geometry}

% % % % % % % % % % % % % % % % % % % % % % % %

% Extra symbols added
\usepackage{stix}
%Three part table - I think this is allowed
\usepackage[flushleft]{threeparttable}

% % % % % % % % % % % % % % % % % % % % % % % %


% amsmath and amssymb packages, useful for mathematical formulas and symbols
\usepackage{amsmath} %amssymb

% Use adjustwidth environment to exceed column width (see example table in text)
\usepackage{changepage}

% Use Unicode characters when possible
\usepackage[utf8x]{inputenc}

% textcomp package and marvosym package for additional characters
\usepackage{textcomp,marvosym}

% cite package, to clean up citations in the main text. Do not remove.
\usepackage{cite}

% Use nameref to cite supporting information files (see Supporting Information section for more info)
\usepackage{nameref,hyperref}

% line numbers
\usepackage[right]{lineno}

% ligatures disabled
\usepackage{microtype}
\DisableLigatures[f]{encoding = *, family = * }

% color can be used to apply background shading to table cells only
\usepackage[table]{xcolor}

% array package and thick rules for tables
\usepackage{array}

% create "+" rule type for thick vertical lines
\newcolumntype{+}{!{\vrule width 2pt}}

% create \thickcline for thick horizontal lines of variable length
\newlength\savedwidth
\newcommand\thickcline[1]{%
  \noalign{\global\savedwidth\arrayrulewidth\global\arrayrulewidth 2pt}%
  \cline{#1}%
  \noalign{\vskip\arrayrulewidth}%
  \noalign{\global\arrayrulewidth\savedwidth}%
}

% \thickhline command for thick horizontal lines that span the table
\newcommand\thickhline{\noalign{\global\savedwidth\arrayrulewidth\global\arrayrulewidth 2pt}%
\hline
\noalign{\global\arrayrulewidth\savedwidth}}


% Remove comment for double spacing
%\usepackage{setspace} 
%\doublespacing

% Text layout
\raggedright
\setlength{\parindent}{0.5cm}
\textwidth 5.25in 
\textheight 8.75in

% Bold the 'Fig #' in the caption and separate it from the title/caption with a period
% Captions will be left justified
\usepackage[aboveskip=1pt,labelfont=bf,labelsep=period,justification=raggedright,singlelinecheck=off]{caption}
\renewcommand{\figurename}{Fig}

% Use the PLoS provided BiBTeX style
\bibliographystyle{plos2015}

% Remove brackets from numbering in List of References
\makeatletter
\renewcommand{\@biblabel}[1]{\quad#1.}
\makeatother

% Leave date blank
\date{}

% Header and Footer with logo
\usepackage{lastpage,fancyhdr,graphicx}
\usepackage{epstopdf}
\pagestyle{myheadings}
\pagestyle{fancy}
\fancyhf{}
\setlength{\headheight}{27.023pt}
\lhead{\includegraphics[width=2.0in]{PLOS-submission.eps}}
\rfoot{\thepage/\pageref{LastPage}}
\renewcommand{\footrule}{\hrule height 2pt \vspace{2mm}}
\fancyheadoffset[L]{2.25in}
\fancyfootoffset[L]{2.25in}
\lfoot{\sf PLOS}
%% Include all macros below

\newcommand{\lorem}{{\bf LOREM}}
\newcommand{\ipsum}{{\bf IPSUM}}

%reference equations in appendices
\usepackage{xr}
\externaldocument{S1Appendix_revised}
\externaldocument{S2Appendix_revised}
\externaldocument{main_revised}

%% END MACROS SECTION

\begin{document}
\vspace*{0.2in}

\setcounter{equation}{0}
\renewcommand{\theequation}{S3.\arabic{equation}}

\section*{S3 Appendix}

\subsection*{Invasion conditions}

A rare sex-determining allele, $m$, will spread in an XY sex-determining system when the leading eigenvalue, $\lambda_m^{(XY)}$, of the Jacobian matrix derived from the eight mutant recursion equations (given by \ref{eq:recursions}c,d,g,h), evaluated at the ancestral equilibrium, is greater than one.
Because a neo-Y (neo-W) is always in males (females) and is epistatically dominant to the ancestral sex-determining locus, the characteristic polynomial factors into two quadratics times $(\lambda_m^{(XY)})^4$, which greatly simplifies analysis.  
One quadratic governs the change in the frequency of the new sex determination factor associated with the $A$ and $a$ allele frequencies, summing across backgrounds at the old sex-determining locus (i.e., summing haplotypes bearing the original X and Y alleles). 
%It is the largest root of this quadratic that determines the spread of the new $m$ allele (see details in \nameref{file:Mathematica}).  
The second quadratic describes the dynamics of the difference in the $m$-$A$ and $m$-$a$ haplotypes between X and Y backgrounds.  
Asymptotically, this difference is constrained to grow at a rate equal to or less than the sum while the $m$ allele is rare, because the frequencies can never become negative. 
We therefore focus on the leading eigenvalue, $\lambda_m^{(XY)}$, which is the largest root of the quadratic $f(x) = x^2+ b x + c = 0$,  where $b= - (\Lambda_{mA}^{(XY)} + \Lambda_{ma}^{(XY)})+(\chi_{mA}^{(XY)} + \chi_{ma}^{(XY)})$ and $c = (\Lambda_{mA}^{(XY)} - \chi_{mA}^{(XY)}) (\Lambda_{ma}^{(XY)} - \chi_{ma}^{(XY)}) - \chi_{mA}^{(XY)} \chi_{ma}^{(XY)}$ (details in \nameref{file:Mathematica}), see Table \ref{tab:haplotype_growth}.  

When $R=0$ the two roots are $\Lambda_{mA}^{(XY)}$ and $\Lambda_{ma}^{(XY)}$, and the leading eigenvalue is the larger of the two.
When $R>0$ then $f(\Lambda_{mA}^{(XY)})$ and $f(\Lambda_{ma}^{(XY)})$ are of opposite signs, and the leading eigenvalue must fall between these two quantities (details in \nameref{file:Mathematica}). %(the Perron-Frobenius theorem guarantees that the largest eigenvalue is positive, unique, and real).  
Thus, $\lambda_m^{(XY)}>1$ if both $\Lambda_{mA}^{(XY)}>1$ and $\Lambda_{ma}^{(XY)}>1$; similarly, $\lambda_m^{(XY)}<1$ if both $\Lambda_{mA}^{(XY)}<1$ and $\Lambda_{ma}^{(XY)}<1$.
If only one haplotypic growth rate is greater than one ($\Lambda_{ma}^{(XY)}<1<\Lambda_{mA}^{(XY)}$ or $\Lambda_{mA}^{(XY)}<1<\Lambda_{ma}^{(XY)}$), then $\lambda_m^{(XY)}$ is greater than one when condition \eqref{eq:lambdasGen} is met. 
Thus, the invasion of a new sex-determining allele is determined by the haplotypic growth rates ($\Lambda_{mj}^{(XY)}$ terms), which do not account for loss due to recombination, and the dissociative force that breaks apart these haplotypes by recombination ($\chi_{mj}^{(XY)}$). 
For tight linkage between the ancestral sex-determining locus ($\mathbf{X}$) and the selected locus ($\mathbf{A}$) we can calculate these terms explicitly (see below).
For weak selection we approximate the leading eigenvalue with a Taylor series. 
The leading eigenvalue for any $k$ is given up to order $\epsilon^2$ by Eq \eqref{eq:lambda_ESD_k}.

\subsubsection*{Tight linkage between $\mathbf{A}$ and $\mathbf{X}$ ($r \approx 0$)}

Here, we explore the conditions under which a neo-W invades an XY system assuming that the $\mathbf{A}$ locus is initially in tight linkage with the ancestral sex-determining locus ($r \approx 0$). 
We disregard neo-Y mutations, which never spread given that the ancestral population is at a stable equilibrium (see \nameref{file:Mathematica} for proof). 

Starting with the simpler $(B)$ equilibrium, the haplotypic growth rates ($\Lambda_{mi}^{(XY)}$) and dissociative forces ($\chi_{mi}^{(XY)}$) are

\begin{subequations}\label{Binvasion}
\begin{align}
\Lambda_{W'A}^{(XY)}&= {\left[w_{A}^\male (1+\alpha^\male_\Delta) \right]}^{-1}
\frac{w_{A}^\female }{w_{A}^\female}
\frac{\left[ w_{A}^\male (1+\alpha^\male_\Delta) w_{AA}^\female + 
w_{a}^\male (1-\alpha^\male_\Delta) w_{Aa}^\female (1+\alpha^\female_\Delta) \right]}
{ 2w_{AA}^\female } \\
\Lambda_{W'a}^{(XY)}&= {\left[w_{A}^\male (1+\alpha^\male_\Delta) \right]}^{-1}
\frac{w_{a}^\female}{w_{A}^\female}
\frac{\left[  w_{A}^\male (1+\alpha^\male_\Delta) w_{Aa}^\female (1-\alpha^\female_\Delta)+
w_{a}^\male (1-\alpha^\male_\Delta) w_{aa}^\female \right]}
{ 2w_{AA}^\female } \\
\chi_{W'A}^{(XY)}&= \frac{1}{2} {\left[w_{A}^\male (1+\alpha^\male_\Delta) \right]}^{-1}
\frac{w_{A}^\female }{w_{A}^\female}
 \frac{\left[  w_{a}^\male (1-\alpha^\male_\Delta) w_{Aa}^\female (1 + \alpha^\female_\Delta) \right]}
{ w_{AA}^\female } \frac{R}{2}\\
\chi_{W'a}^{(XY)}&= \frac{1}{2} {\left[w_{A}^\male (1+\alpha^\male_\Delta) \right]}^{-1}
\frac{w_{a}^\female}{w_{A}^\female}
 \frac{\left[  w_{A}^\male (1+\alpha^\male_\Delta) w_{Aa}^\female (1-\alpha^\female_\Delta) \right]}
{  w_{AA}^\female} \frac{R}{2}.
\end{align}
\end{subequations}

%\noindent
%In this case, the zygotic sex ratio ($\zeta$) is given by the difference in haploid selection in males on $a$ (fixed on the Y) and $A$ (fixed on the X) alleles, i.e., there are more males than females if $\zeta = \alpha^\male w_A^\male / [(1-\alpha^\male)w_a^\male + \alpha^\male w_A^\male]<1/2$. %$\alpha^\male<1/2$ and/or $w_{a}^\male>w_{A}^\male$. 
%Populations with haploid selection for $a$ in males have a male biased zygotic sex ratio are thus more permissive to invasion by a neo-W ($\Lambda_{mA}$ and $\Lambda_{ma}$ larger).
%%In addition, the spread of neo-W alleles depends on the fitness of female diploids after mating with an X-$A$ or Y-$a$ male gamete, relative to the fitness of females in the population (all have genotype $AA$ and fitness $w_{AA}^\female$ at this equilibrium). 
%Haploid selection in males has a second effect; the spread rate of neo-W haplotypes is determined by their fitness in diploid females, which depends on their diploid genotype and thus on the male gamete they pair with. 
%Zygotes carrying dominant neo-W alleles will develop as females regardless of their genotype at the XY locus. 
%Therefore, neo-W females result from matings with either X-$A$ or Y-$a$ male gametes. 
%The relative proportion of these male gametes is determined by haploid selection in males; mating with a Y-$a$ male gamete is more likely if the $a$ allele is favoured during male gamete production or competition ($\zeta<1/2$). 
%Thus, neo-W females experience different diploid selection than XX females, and the extent of this difference depends on haploid selection in males.
%Furthermore, haploid selection in females can directly select upon neo-W-$A$ or neo-W-$a$ haplotypes. 
%A neo-W-$A$ female gamete has the same fitness during haploid competition as resident $A$-bearing female gametes. 
%On the other hand, neo-W-$a$ female gametes can be favoured or disfavoured during female haploid competition (favoured if $w_{a}^\female>w_{A}^\female$). 
%Meiotic drive in females ($\alpha^\female$) similarly affects the fitness of these neo-W haplotypes, except that it impacts both haplotypes as meiotic drive only occurs in heterozygotes and therefore does not occur in resident XX females (who are always homozygote $AA$). 

Haploid selection impacts the spread of neo-W haplotypes in three ways (also seen in Table \ref{tab:haplotype_growth}).
Firstly, the zygotic sex ratio becomes male biased, $\zeta>1/2$, when the $a$ allele (which is fixed on the Y) is favoured during competition among male gametes or by meiotic drive in males.
Specifically, at equilibrium $(B)$, female zygote frequency is $1 - \zeta = w_A^\male (1+\alpha^\male_\Delta) / (2\bar{w}_{H}^\male)$ where $2\bar{w}_{H}^\male= \big[w_a^\male (1-\alpha^\male_\Delta) +  w_A^\male (1+\alpha^\male_\Delta) \big]$ has been canceled out in equations \eqref{Binvasion} to leave the term ${\left[w_{A}^\male (1+\alpha^\male_\Delta) \right]}^{-1}$. 
Male biased sex ratios facilitate the spread of a neo-W because neo-W alleles cause the zygotes that carry them to develop as the rarer, female, sex. 

Secondly, haploid selection in females selects on neo-W haplotypes directly.
At equilibrium $(B)$, the fitness of female gametes under the ancestral sex-determining system is $w_{A}^\female$ such that the relative fitnesses of neo-W-$A$ and neo-W-$a$ haplotypes during female gametic competition are $w_{A}^\female/w_{A}^\female$ and $w_{a}^\female/w_{A}^\female$ (see terms in equation \ref{Binvasion}). 
Meiotic drive in females will also change the proportion of gametes that carry the $A$ versus $a$ alleles, which will be produced by heterozygous females in proportions $(1+\alpha_{\Delta}^\female)/2$ and $(1-\alpha_{\Delta}^\female)/2$, respectively. 
These terms are only associated with heterozygous females, i.e., they are found alongside $w_{Aa}^\female$.

Thirdly, haploid selection in males affects the diploid genotypes of females by altering the allele frequencies in the male gametes with which female gametes pair.
At equlibrium $(B)$, neo-W female gametes will mate with X-$A$ male gametes with probability $ w_A^\male (1+\alpha^\male_\Delta) / (2\bar{w}_{H}^\male)$ and Y-$a$ male gametes with probability $w_a^\male (1-\alpha^\male_\Delta) / (2\bar{w}_{H}^\male)$, where the $2\bar{w}_{H}^\male$ terms have been canceled in Eq \eqref{Binvasion} (as mentioned above). 
Thus, neo-W-$A$ haplotypes are found in $AA$ female diploids with probability $ w_A^\male (1+\alpha^\male_\Delta)/ (2\bar{w}_{H}^\male)$ (e.g., first term in square brackets in the numerator of equation \ref{Binvasion}a) and in $Aa$ female diploids with probability $w_a^\male (1-\alpha^\male_\Delta) / (2\bar{w}_{H}^\male)$ (e.g., the second term in square brackets in the numerator of equation \ref{Binvasion}a).

%For instance, because an epistatically dominant neo-W always causes its carrier to become female, it creates females who carry either Y-$a$ or X-$A$ genotypes from their father.
%Thus, because when there is a polymorphism the X carries some non-zero frequency of $A$, haploid selection in males impacts the diploid genotypes of females (e.g., creating more $Aa$ females when drive in males favours Y-$a$).
%How this affects the spread of the neo-W then depends on diploid and haploid selection in females.

The other terms in equations \eqref{Binvasion} are more easily interpreted if we assume that there is no haploid selection in either sex, in which case $\Lambda_{W'A}^{(XY)}=(w_{AA}^\female+w_{Aa}^\female)/2 w_{AA}^\female$ and $\Lambda_{W'a}^{(XY)}=(w_{aa}^\female+w_{Aa}^\female)/ 2 w_{AA}^\female$.
Neither haplotype can spread under purely sexually-antagonistic selection at equilibrium $(B)$, with directional selection in each sex.
Essentially, the X is then already as specialized as possible for the female beneficial allele ($A$ is fixed on the X background), and the neo-W often makes daughters with the Y-$a$ haplotype, increasing the flow of $a$ alleles into females, which reduces the fitness of those females.  

If selection doesn't uniformly favour $A$ in females, however, neo-W-$A$ haplotypes and/or neo-W-$a$ haplotypes can spread ($\Lambda_{W'A}^{(XY)}>1$ and/or $\Lambda_{W'a}^{(XY)}>1$).
%If $\Lambda_{mA}>1$ (requiring $w_{Aa}^\female>w_{AA}^\female$), the implication is that the neo-W can spread alongside the $A$ allele, despite the fact that the neo-W sometimes brings $Y-a$ haplotypes into females, because $a$ is favoured by selection in females despite $A$ being fixed on the X.
A neo-W-$A$ haplotype can spread ($\Lambda_{W'A}^{(XY)}>1$) when $w_{Aa}^\female>w_{AA}^\female$, despite the fact that a neo-W brings Y-$a$ haplotypes into females.
In this case the $a$ allele is favoured by selection in females despite $A$ being fixed on the X background.
For this equilibrium to be stable (i.e.,  to keep $A$ fixed on the X), X-$a$ cannot be overly favoured in females and X-$A$ must be sufficiently favoured in males (for example, by overdominance in males). 
Specifically, from the stability conditions for equilibrium (B), we must have $w_{Aa}^\female < 2 w_{AA}^\female$ and $w_{Aa}^\male/\big[(w_{aa}^\male+w_{Aa}^\male)/2\big]>w_{Aa}^\female/w_{AA}^\female$. 

Still considering $w_{Aa}^\female>w_{AA}^\female$, the neo-W can also spread alongside the $a$ allele ($\Lambda_{W'a}^{(XY)}>1$) if $w_{aa}^\female$ is large enough such that $(w_{Aa}^\female+w_{aa}^\female)/2>w_{AA}^\female$.
%there is sufficiently strong underdominance in females ($w_{aa}^\female>w_{Aa}^\female$) \textcolor{blue}{[this is describing directional selection, not overdominance - check the conditions for the case we want to talk about]}, such that $(w_{Aa}^\female+w_{aa}^\female)/2>w_{AA}^\female$.  
This can occur with overdominance or directional selection for $a$ in females (Fig \ref{fig:regionplots}B,C).
In this case, $a$ is favoured on the ancestral Y background in males and on the ancestral X background in females (comparing $Aa$ to $AA$ genotypes in females) but $A$ is fixed on the X background due to selection in males. 
The neo-W-$a$ haplotype can spread because it produces females with higher fitness $Aa$ and $aa$ genotypes. 

%When both haplotypes can spread on their own ($\Lambda_{mA}>1$ and $\Lambda_{ma}>1$), the neo-W invades regardless the recombination rate between it and the selected locus, $R$.
%When neither haplotype can spread ($\Lambda_{mA}<1$ and $\Lambda_{ma}<1$) the neo-W can never invade.
%And when only one haplotype can spread on its own the neo-W invades only when the rate of recombination onto the favourable background is sufficiently larger than the rate of recombination off this background (i.e., equation\ref{eq:lambdasGen} is satisfied).

%Haploid selection affects the conclusions outlined above by altering the zygotic sex ratio and exerting direct selection on neo-W haplotypes.
%The effect of sex ratio selection is straightforward and can be seen in Eq \eqref{Binvasion}, where male drive and gamete competition for $a$ (which is fixed on the Y and therefore causes a male bias) increases all four terms.
%Direct selection on neo-W haplotypes can also be seen in Eq \eqref{Binvasion} by noticing that increasing female drive in favour of $A$ increases the neo-W-$A$ haplotype growth rate ($\Lambda_{mA}$) while female drive and gamete competition in favour of $a$ increases the neo-W-$a$ haplotype growth rate ($\Lambda_{ma}$).
%Gamete competition in females does not impact the spread of neo-W-$A$ haplotypes because they are on the same background as the resident X.
%Recombination between the selected locus and the neo-W complicates things slightly, as selective advantages for one neo-W haplotype during the haploid stage in females comes at a cost to the other, and thus recombination off the favoured background becomes more detrimental.

%\textcolor{red}{the versions for equil $(A)$ are not terrible but could be improved by cleverly re-writing with the $\phi$ and $\psi$ terms from above, I think...}

Similar equations can be derived for equilibrium (A) by substituting the equilibrium frequencies into Table \ref{tab:haplotype_growth}

\begin{subequations}\label{Ainvasion}
\begin{align}
\Lambda_{W'A}^{(XY)}&= \frac{a}{b} \left[w_{AA}^\female w_{Aa}^\male w_A^\male (1+\alpha^\male_\Delta) \phi +  w_{Aa}^\female (1+\alpha^\female_\Delta) w_a^\male c \right] / (2 w_a^\female) \\
\Lambda_{W'a}^{(XY)}&= \frac{a}{b} \left[ w_{Aa}^\female (1-\alpha^\female_\Delta) w_{Aa}^\male w_A^\male (1+\alpha^\male_\Delta) \phi +  w_{aa}^\female w_a^\male c \right] / (2 w_A^\female) \\
\chi_{W'A}^{(XY)}&= \frac{a}{b} \frac{R}{2} \left[   w_{Aa}^\female (1+\alpha^\female_\Delta) w_a^\male c \right] / w_a^\female\\ 
\chi_{W'a}^{(XY)}&= \frac{a}{b} \frac{R}{2} \left[ w_{Aa}^\female (1-\alpha^\female_\Delta) w_{Aa}^\male w_A^\male (1+\alpha^\male_\Delta) \phi \right] / w_A^\female,
%\Lambda_{W'A}^{(XY)}&=   {\left[ 2 (1 - \zeta) \right]}^{-1} w_A^{\female} \left[ b w_{AA}^\female + c w_{Aa}^\female \left(1 - \alpha^\female_\Delta \right) \right]/ \left(\bar{w}_H^\female \bar{w}_H^\male \bar{w}^{\female}_{D} \right)  \\
%\Lambda_{W'a}^{(XY)}&=  \\
%\chi_{W'A}^{(XY)}&= \\ 
%\chi_{W'a}^{(XY)}&=
\end{align}
\end{subequations}

\noindent
where 

\begin{subequations}
\begin{align}
a &= w_a^\female \phi + w_A^\female \psi \\
b &= w_{AA}^\female \left[ w_{Aa}^\male w_A^\male (1+\alpha^\male_\Delta) \right] \phi^2 + w_{Aa}^\female \left[ w_{Aa}^\male w_A^\male (1+\alpha^\male_\Delta) + w_{aa}^\male w_a^\male \right] \psi \phi + w_{aa}^\female \left( w_{aa}^\male w_a^\male \right) \psi^2\\
%c &= w_{Aa}^\male (1-\alpha^\male_\Delta) \phi \left[ w_{Aa}^\male (1+\alpha^\male_\Delta) \phi \right] + w_{Aa}^\male\phi \left[ 2w_{AA}^\male + w_{aa}^\male (1+\alpha^\male_\Delta) \right] \psi + w_{aa}^\male \psi \left(2w_{AA}^\male \psi \right) \\
c &= w_{Aa}^\male (1-\alpha^\male_\Delta) \phi + 2w_{aa}^\male \psi.
\end{align}
\end{subequations}

%\textcolor{red}{Sally suggests using some $\zeta$ term in the above equations}
%\textcolor{blue}{Maybe just reference table 2?}

As with equilibrium (B), haploid selection again modifies invasion fitnesses by altering the ancestral sex ratio, $\zeta$ (see Table \ref{tab:haplotype_growth}), and directly selecting upon female gametes, through $w_{i}^\female$.
The only difference is that resident XX females are no longer always homozygote $AA$ and males are no longer always heterozygote $Aa$.
Thus the effect of haploid selection in males is reduced, as is the difference in fitness between neo-W haplotypes and resident X haplotypes, as both can be on any diploid or haploid background.  

The other terms are easier to interpret in the absence of haploid selection.
For instance, without haploid selection, the neo-W-$A$ haplotype spreads ($\Lambda_{W'A}^{(XY)}>1$) if and only if

\begin{equation}\label{eq:BeqWAspread}
2(w_{Aa}^\female-w_{aa}^\female)w_{aa}^\male \psi^2 > (w_{AA}^\female-w_{Aa}^\female)w_{Aa}^\male \phi (\phi-\psi),
\end{equation}

\noindent  
where $\phi-\psi=w_{AA}^\female w_{Aa}^\male-w_{aa}^\female w_{aa}^\male$ and both $\phi$ and $\psi$ are positive when equilibrium (A) is stable. 
In contrast to equilibrium (B), a neo-W haplotype can spread under purely sexually-antagonistic selection  ($w_{aa}^\female<w_{Aa}^\female<w_{AA}^\female$ and $w_{AA}^\male<w_{Aa}^\male<w_{aa}^\male$).
The neo-W-$A$ can spread as long as it becomes associated with females that bear more $A$ alleles than observed at equilibrium (A), effectively specializing on female fitness. 
%In this case, the neo-W-$A$ haplotype can spread, despite producing a lot of $Aa$ daughters by obtaining the $a$ from Y-gametes, when $aa$ females, which the neo-W-$A$ never makes, are strongly selected against.
%This can be intuited from the fact that \eqref{eq:BeqWAspread} will be more easily met when $w_{Aa}^\female-w_{aa}^\female\approx w_{Aa}^\female$ and $w_{AA}^\female-w_{Aa}^\female\approx 0$, implying $w_{aa}^\female \approx 0$ and $w_{Aa}^\female\approx w_{AA}^\female$ (although this is complicated by the fact that $w_{aa}^\female$ and $w_{Aa}^\female$ affect $\phi$ and $\psi$ too, the intuition holds). 

Without haploid selection, the neo-W-$a$ haplotype spreads ($\Lambda_{W'a}^{(XY)}>1$) if and only if

\begin{equation}\label{eq:BeqWaspread}
(w_{aa}^\female + w_{Aa}^\female-2w_{AA}^\female)w_{Aa}^\male \phi^2 + (w_{aa}^\female-w_{Aa}^\female)(w_{Aa}^\male+2w_{aa}^\male) \phi \psi >0.
\end{equation}

\noindent
This condition cannot be met with purely sexually antagonistic selection (as both terms on the left-hand side would then be negative), but it can be met under other circumstances. 
For example, with overdominance in males there is selection for increased $A$ frequencies on the X background in males, which are always paired with Y-$a$ haplotypes.
Directional selection for $a$ in females can then maintain a polymorphism at the $\mathbf{A}$ locus on the X background.
This scenario selects for a modifier that increases recombination between the sex chromosomes (e.g., blue region of Fig 2d in~\cite{Otto2014}) and facilitates the spread of neo-W-$a$ haplotypes, which create females bearing more $a$ alleles than the ancestral X haplotype does. 

In absence of haploid selection, the fact that a less closely linked neo-W ($R>0$) can invade an XY system with tight sex-linkage can also be reached from Equation 7 in \cite{vanDoorn:2010hu}; for example, with no polymorphism on the Y ($V_Y=0$) and an allelic substitution favoured in females ($\alpha^f,\alpha_X^f>0$) a loosely linked neo-W can invade given the allelic substitution is sufficiently disfavoured on the X in males ($\alpha_X^m < -2\alpha_X^f$), although it is unclear from their implicit equation if and when such an equilibrium is stable.

\subsubsection*{Role of haploid selection with tight linkage between $\mathbf{X}$ and $\mathbf{A}$ loci}

Haploid selection generally expands the conditions under which neo-W alleles can spread within ancestral systems that have evolved tight linkage between the sex-determining locus and a selected locus ($r \approx 0$).
First, haploid selection can allow a polymorphism to be maintained when it would not under diploid selection alone (e.g., with directional selection in diploids). 
In cases of ploidally-antagonistic selection, where there is a balance between alleles favored in the haploid stage and the diploid stage, neo-W alleles - even if unlinked to the selected locus - can spread (\nameref{fig:regionPloidAntag}).
Second, even when diploid selection could itself maintain a polymorphism, haploid selection can increase the conditions under which transitions among sex-determining systems are possible.
Of particularly importance, when selection is sexually-antagonistic in diploids ($s^\female s^\male <0$ and $0<h^\circ<1$), an unlinked neo-W ($R=1/2$) cannot invade unless there is also haploid selection (see proof in \nameref{file:Mathematica}; Fig \ref{fig:SexAntagTighter} and \nameref{fig:SexAntagTighterMaleDrive}). 
More generally, haploid selection alters the conditions under which neo-W alleles can spread (compare \nameref{fig:regionMaleDrive}-\nameref{fig:regionFemaleGS} to Fig \ref{fig:regionplots}). 

%[SALLY:  I think that is enough, but we could add somewhere??: "Without haploid selection, unlinked neo-W alleles ($R=1/2$) can invade only under certain conditions, requiring either overdominance in males or underdominance in females (see proof in supplementary \textit{Mathematica} file).?  Note that underdominance in females with directional selection in males can lead to the spread of unlinked neo-W?s, but only from equilB (only $\Lambda_{ma}$ is greater than one in this case).]

Male haploid selection in favour of the $a$ allele ($\alpha_{\Delta}^\male<0$, $w_{A}^\male<w_{a}^\male$) generates male-biased sex ratios at equilibria (A) and (B), where Y-$a$ is fixed ($\hat{p}_{Y}^\male=0$). 
Male-biased sex ratios facilitate the spread of neo-W-$A$ and neo-W-$a$ haplotypes (increasing $\Lambda_{W'A}^{(XY)}$ and $\Lambda_{W'a}^{(XY)}$). 
Panels A-C in \nameref{fig:regionMaleDrive} and \ref{fig:regionMaleGS} show that neo-W haplotypes tend to spread for a wider range of parameters when sex ratios are male biased, compared to Fig \ref{fig:regionplots} without haploid selection. 
By contrast, male haploid selection in favour of the $A$ allele generates female-biased sex ratios and reduces $\Lambda_{W'A}^{(XY)}$ and $\Lambda_{W'a}^{(XY)}$, as demonstrated by panels D-F in \nameref{fig:regionMaleDrive} and \nameref{fig:regionMaleGS}. 

Female haploid selection generates direct selection on the neo-W-$A$ and neo-W-$a$ haplotypes as they spread in females. 
Thus, female haploid selection in favour of the $a$ allele tends to increase $\Lambda_{W'a}^{(XY)}$ and decrease $\Lambda_{W'A}^{(XY)}$, as shown by panels A-C in \nameref{fig:regionFemaleDrive} and \nameref{fig:regionFemaleGS}. 
Conversely, female haploid selection in favour of the $A$ allele increases $\Lambda_{W'A}^{(XY)}$ and decreases $\Lambda_{W'a}^{(XY)}$, see panels D-F in \nameref{fig:regionFemaleDrive} and \nameref{fig:regionFemaleGS}. 

Thus, the impact of haploid selection on transitions between sex-determining systems must be considered as two sides of a coin: it can generate sex ratio biases that promote transitions that equalize the sex ratio, but it can also direct select for transitions that cause sex ratios to become biased.

\bibliography{sex_chromosomes.bib}

\end{document}