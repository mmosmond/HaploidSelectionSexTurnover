\documentclass[12pt]{article}
%\documentclass{nature}

% Including pdf figures
\usepackage{graphicx}
\graphicspath{ {../Plots/} }
\usepackage{pdfpages}
\usepackage{epstopdf}
\epstopdfsetup{outdir=./}
%really place a figure in a location
\usepackage{float}
%Overrun caption
\usepackage[CaptionAfterwards]{fltpage}
% Math stuff
\usepackage{amsmath}
\usepackage{stix}
% Bibliographies
\usepackage[numbers]{natbib}
\bibpunct{(}{)}{,}{a}{}{;} 

\usepackage[flushleft]{threeparttable}

%\usepackage[font={scriptsize}]{caption}
\usepackage[labelfont=bf,labelsep=period]{caption}

\usepackage{lineno} %gives line numbers with \lineno command

\usepackage{setspace}
\onehalfspace

\usepackage{tikz}%for putting words on figures
\usetikzlibrary{positioning}%for relative positioning

%to rotate figures
\usepackage{rotating}
\usepackage{pdflscape}

\begin{document}

\setcounter{table}{0}
\renewcommand{\thetable}{S\arabic{table}}

\begin{table}[ht]
\centering
\smallskip
\caption{
Substitutions for different loci orders assuming no interference. %and $r,R\in(0,1/2)$.\textcolor{blue}{write all in form of first line? so that 1/2 cases are okay (can't determine chi if R is 1/2 in second line, or if r is 1/2 in third line}
}
\begin{tabular}{l l}
\hline\hline
  Order of loci & Substitution  \\ [0.5ex] \hline
  SDR-A-M & $\rho=r(1-R)+R(1-r)$  \\
  SDR-M-A & $r=\rho(1-R)+R(1-\rho)$ \\ %$\rho=(r-R)/(1-2R)$ \\
  A-SDR-M & $R=r(1-\rho)+\rho(1-r)$ \\ %$\rho=(R-r)/(1-2r)$ \\
  \hline \hline
  \label{tab:chisubstitutions}
 \end{tabular}
\end{table}

\end{document}



