\documentclass[12pt]{article}
%\documentclass{nature}

% Including pdf figures
\usepackage{graphicx}
\usepackage{pdfpages}
%really place a figure in a location
\usepackage{float}
%Overrun caption
\usepackage[CaptionAfterwards]{fltpage}
% Math stuff
\usepackage{amsmath}
% Bibliographies
\usepackage[numbers]{natbib}
\bibpunct{(}{)}{,}{a}{}{;} 

\usepackage[font={scriptsize}]{caption}

\usepackage{lineno} %gives line numbers with \lineno command

\usepackage{setspace}
\onehalfspace

\begin{document}

\title{Gametic Selection, Sex Ratio Bias, and Transitions Between Sex Determination Systems}
\author{Michael F Scott$*^1$ and Matthew M Osmond$*^2$, and Sarah P Otto$^2$}
\date{}
\maketitle
\noindent
$*$ These authors contributed equally to this work

\noindent
$^1$ Department of Botany, University of British Columbia, \#3529 - 6270 University Boulevard, Vancouver, BC, Canada V6T 1Z4

\noindent
$^2$ Department of Zoology, University of British Columbia, \#4200 - 6270 University Boulevard, Vancouver, BC, Canada V6T 1Z4

\noindent
email: mfscott@biodiversity.ubc.ca, mmosmond@zoology.ubc.ca

\noindent
Contributions: 

\newpage

\begin{abstract}
Sex determination systems are remarkably dynamic; many studied taxa display transitions of sex-determining genes between chromosomes or the evolution of new sex-determining systems. 
%Predominant theories in which new sex-determining systems are favoured by selection involve sex ratio selection or fitness differences between sexes (e.g., sexually antagonistic selection). 
Here, we utilize population genetic models to study the spread of novel sex-determining systems where we also include 
%a period of selection during the haploid stage produced by one sex, e.g., pollen or sperm competition. 
haploid gametic selection, e.g., pollen or sperm competition.
Haploid selected loci experience a form of sex-specific selection (because gametic competition occurs predominantly among haploids produced by males) and can also cause sex ratios at birth to become biased (because sex ratios are determined by the fertilization success of X- versus Y-bearing pollen/sperm). 
We find that the evolution of sex determination systems where mothers determine sex at birth (e.g., environmental sex determination where sex is determined at birth) is influenced by classic Fisherian sex ratio selection. {\color{red}(Maybe not true???)}
However, notably, we find that the spread of new genetic sex determination systems is not affected by sex ratio biases that are caused by gametic selection because sex ratios become biased after parental provisioning has occurred (even if pollen/sperm competition occurs within the mother). 
In addition, we find that linkage of an ancestral sex chromosome to a locus under haploid selection can favour transitions between male and female heterogamety (e.g., XY to ZW), which is not the case for any forms of diploid sex specific selection (e.g., sexually antagonistic selection).
During these transitions, new sex-determining alleles spread despite breaking up favourable associations that build up between ancestral sex-determining loci and selected loci, reducing population mean fitness. 
Furthermore, a period of selection among haploids can favour the stable maintenance of polymorphic sex determination systems. 
Thus, our models offer several new insights to be explored as information about sex determination in non-model taxa accumulates.

%[Approx 320 words]
\end{abstract}

\newpage

\section*{Introduction}

%CHECK THROUGHOUT: use of sex-determination SYSTEMS versus MECHANISMS. 

Animals and angiosperms exhibit extremely diverse sex determination systems, as reviewed by \citet{Bull:1983vi,Charlesworth:2010it,Beukeboom:2014vb,Bachtrog:2014bx}. 
Among species with genetic sex determination of diploid sexes, some taxa have heterogametic males (XY) and homogametic females (XX), including %non-monotreme? 
mammals and most dioecious plants \citep{Ming:2011iy}; whereas other taxa have homogametic males (ZZ) and heterogametic females (ZW), including Lepidoptera and birds. 
Within several taxa, the chromosome that harbours the master sex-determining region changes. 
For example, transitions of the master sex-determining gene between chromosomes or the evolution of new master sex-determining genes have occurred in Salmonids \citep{Li:2011fm,Yano:2012di}, Diptera \citep{Vicoso:2015hf}, and \textit{Oryzias} \citep{Myosho:2012fv}.
%Presentation at evolution found neo sex chromosome in birds, doesn't seem to be published or bioRxiv yet (consider pers com.), title: A previously unknown neo-sex chromosome mediates plumage divergence and speciation in hybridizing birds Jason Sardell; Elizabeth Cooper; J. Albert C. Uy. I wonder if this is actually a fusion event though?
In addition, many gonochoric/dioecious clades with genetic sex determination exhibit transitions between male (XY) and female (ZW) heterogamety, including eight of 26 teleost fish families \citep{Mank:2006bt}, true fruit flies \citep[Tephritids,][]{Vicoso:2015hf}, amphibians \citep{Hillis:1990gu}, the angiosperm genus \textit{Silene} \citep{Slancarova:2013dq}, Coleoptera and Hemiptera \citep[][plate 2]{Beukeboom:2014vb}.
Indeed, in some cases, both male and female heterogametic sex determination systems can be found the same species, including cichlid species \citep{Ser:2010iq} and \textit{Rana rugosa} \citep{Ogata:2007jm}.

\noindent
Depending on the prominence of transitions to ESD in the manuscript include something like (currently quoted):

``Transitions have repeatedly occurred between environmental sex determination and genotypic sex determination, as exemplified by the distribution of temperature sex determination among reptiles: either temperature or genes provide the initial trigger in closely related species (Ewert and Nelson 1991; Pokorna and Kratochvil 2009; Ezaz et al. 2009) or even conspecific populations (Pen et al. 2010). Similar situations are found in fishes (e.g., Conover and Heins 1978a).''

\noindent
We have results where polygenic sex determination is sometimes stable, may be worth mentioning:

``Polygenic sex determination has been reported in many plants (e.g. Shannon \& Holsinger 2007), fishes (Vandeputte et al. 2007; Ser et al. 2010; Liew et al. 2012), crustaceans (e.g. Battaglia 1958; Battaglia \& Malesani 1959; Voordouw
\& Anholt 2002), bivalves (Haley 1977; Saavedra et al. 1997), gastropods (Yusa 2007a,b), and polychaetes (Bacci 1965, 1978; Premoli et al. 1996).''
From Vuilleumier et al. 2007:
``Polymorphism for sex-determining genes within oramong populations has been reported in many speciesincluding houseflies, midges, woodlice, platyfish, cichlidfish, and frogs (Gordon, 1944; Kallman, 1970; Thomp-son, 1971; Macdonald, 1978; Bull, 1983; Rigaud et al., 1997; Caubet et al., 2000; Lande et al., 2001; Ogataet al., 2003; Lee et al., 2004; Mank et al., 2006).

%NOTE: ``Bull & Charnov (1977) hypothesized that a new sex-determining gene can rapidly increase and become fixedin a population if it is linked to a gene with high adaptivevalue, and finally cause a change of the heterogametic sex.'' The Bull and Charnov paper is also the one in which the set of neutral equilibria between XY and ZW are identified (where sex ratios are equal)
%NOTE: Vuilleumier et al. 2007 also find polymorphic sex determination can be stable.

%For example, closely related species in the Medaka lineage (cite Takenhana et al. 2008) and Tilapiinae (cite Lee et al 2004, Cnaani et al 2008), citations in Beukeboom. 

\noindent
\textcolor{red}{Brief description of sex ratio adjustment and sexual antagonism theories:}

Predominant theories in which new sex determination systems are favoured by selection involve fitness differences between sexes (e.g., sexually antagonistic selection) or sex ratio selection.
\citet{vanDoorn:2007eu,vanDoorn:2010hu} show that new sex determination loci can be favoured if they arise in close linkage with a locus that experiences sexual antagonism. 
For example, linkage allows favourable associations to build up between a male-beneficial allele and a neo-Y chromosome. 
Such associations can favour a new master sex-determining gene on a new chromosome \citep{vanDoorn:2007eu} and can also favour a transition between male and female heterogamety \citep[e.g., a ZW to XY transition,][]{vanDoorn:2010hu}.
However, any sexually-antagonistic loci that are linked to the ancestral sex-determination locus will develop similar, favourable associations and select against the spread of a new sex-determination system. 

Sex ratio selection might be a particularly important force driving transitions between sex-determining systems \citep[Chapter 7]{Beukeboom:2014vb}. 
For example, feminizing mutations may invade when female biased sex ratios are favoured due to interdemic selection \citep{Wilson:1981vm,Vuillleumier:2007bh}.
In other cases, flexible sex determination systems may be favoured in order to exploit environmental conditions that are optimal for males or females, creating locally biased sex ratios \citep{Charnov:1977tx,Werren:1984tl,Pen:2010kk}. 
In other situations, sex ratio selection may favour transitions in order to restore equal sex ratios. 
For example, \citet{Kozielska:2010vm} consider systems in which the ancestral sex chromosomes experience meiotic drive (e.g., driving X or Y chromosomes are inherited disproportionately often), which leads to biased sex ratios. They find that new, unlinked sex-determining loci (masculinizing or feminizing mutations) can then spread, restoring an even sex ratio. 

\noindent
\textcolor{red}{We add haploid selection:}

Here, we use mathematical models to find the conditions under which new sex determination systems are favoured by selection where we include a period of selection among haploid gametes/gametophytes. 
{\color{blue}
FROM PREVIOUS PAPER:
In plants, selection among haploid male gametophytes is thought to be pervasive \cite{SKOGSMYR:2002ce,Moore:2011jt,Marshall:2016fe}; in \textit{Arabidopsis}, 60-70\% of all genes are expressed during the haploid phase \cite{Borg:2009jpa}, and pollen expressed genes exhibit stronger signatures of purifying selection and positive selection \cite{Arunkumar:2013iq,Gossmann:2014dua}.
For agricultural breeding, pollen has been exposed to a variety of selection pressures \textit{in vivo} and \textit{in vitro}, including temperature \cite{Hedhly:2004iv,Clarke:2004ir}, herbicides \cite{Frascaroli:2001ee}, metals \cite{Searcy:1985vt}, water stress \cite{Ravikumar:2003uo}, and pathogens \cite{Ravikumar:2012ej}, resulting in an increased frequency of resistant genotypes among the diploid sporophytic offspring. 
In animals, expression during the haploid sperm stage is traditionally thought to be suppressed \cite{Hecht:1998hz}, although recent evidence suggests that the extent and selective importance of postmeiotic gene expression may be underestimated \cite{Zheng:2001fi,JOSEPH:2004haa,Vibranovski:2010et,Immler:2014im}. 
}

Here, suggest that the canonical view (no haploid expression in animals, genome highly chromatinized in sperm and not expressed) might be based on model organisms, such as mice, where sperm is sufficiently short-lived that transcripts provisioned during spermatogenesis may be sufficient without further haploid transcription (although note that the Vibranovski lab results are in mice showing some transcription does occur). 
In broadcast spawning animal species (e.g., corals, many fish) and species where sperm typically requires greater longevity, expression of the haploid genotype may be required (Immler paper indicates this, but not that strongly - as I remember). 
We can use this suggestion in discussion to speculate in what species the processes we study might be looked for (i.e.., animals with multiple matings, broadcast spawning and/or long-lived sperm and outcrossing/non-pollen-limited plants). 

{\color{blue} 
\noindent FROM PREVIOUS PAPER:

The maintenance of polymorphism at loci that experience sex specific selection in both haploid and diploid phases was considered by Immler et al. \cite{Immler:2012tl}, demonstrating that polymorphisms can be maintained by sexually antagonistic selection or overdominance as well as by conflicting selection pressures in haploids and diploids (haploid-diploid conflict or ploidally antagonistic selection) or a combination of these selective regimes.  

}

We add haploid selection (and justification, see below)
\\
Also discuss the fact that, in terms of recombination suppression, haploid selection among male gametes generates selection pressure similar to that of male specific selection. 
\\
What will be the result: where there is sex biases and sex-specific selection




\textcolor{red}{NOTE RE: DRIVE. I expect drive (that occurs specifically in one sex, e.g., during spermatogenesis) to behave almost exactly like haploid selection. That is, I think that a XY-linked driver that is maintained by selection (e.g., because it causes sterility when homozygous, which is common in known drive systems) will only favour invasion of a more tightly linked neo-Y (worsening sex ratio biases) and could favour invasion of a neo-W. This may run counter to generic expectations from new sex chromosome systems evolving to balance the sex ratio. So, do you think it would significantly enhance the paper to model drive explicitly or just discuss it as being similar???}

\linenumbers
\modulolinenumbers[2]

\section*{Discussion}


DRAFT (improve): In \citet{Ubeda:2015fx}, the new sex determining locus spreads because it arises in linkage with a locus that experiences drive. They assume that drive occurs predominantly in one sex, e.g., during spermatogenesis or a 'killer' sperm. A driving allele is maintained at an intermediate frequency by selection, e.g., because it causes male sterility when homozygous (because all male sperm are killed). Y chromosomes that arise in linkage with the driving allele spread because they allow drive to occur more often, thus genetic sex determination with a sex ratio bias evolves. 
Thus \citet{Ubeda:2015fx} also find that genetic sex determiners can invade, despite causing sex ratios to become biased. 
Finally, they show that autosomal 'restorers' that negate the effects of meiotic drive can invade and restore an equal sex ratio. 
When sex ratio bias occurs due to haploid selection, a natural class of sex ratio `restorers' exist because haploid selection often occurs in a context that is determined by the diploid parents. 
For example, the intensity of pollen competition can be manipulated by altering style length \citep{Travers:2001,Lankinen:2001gc,Ruane:2009vt}, delaying stigma receptivity \citep{Galen:1986wq,Lankinen:2011if} and/or delaying pollen tube growth in the pistil \citep{Herrero:2003jf}. 
Where the X and Y have fitness differences, \citet{Hough:2013uo} and \citet{Otto:2015va} demonstrated that mothers should generally evolve to balance sex ratios by reducing the intensity of haploid competition. 

{\color{blue}
FROM THESIS:
However, reducing competition among haploids also reduces the potential for harmful deleterious mutations to be purged. 
When deleterious mutations are included, the optimal intensity of haploid selection can reflect a balance between maximizing offspring fitness and equalizing sex ratios. 

As part of a collaborative project \citep{Otto:2015va}, I considered the evolution of the haploid `selective arena' in cases where the X chromosome harbours a polymorphism that affects haploid fitness. 
Mothers again primarily evolve to restore equal sex ratios.
However, modifying haploid selection also affects the X-linked genotypes that are inherited by offspring. 
Specifically, increasing the intensity of haploid selection increases the proportion of daughters (all progeny of X-bearing sperm/pollen are female) that inherit the allele with high haploid fitness. 
If this allele has high fitness in daughters, mothers can be selected to increase the intensity of haploid selection; otherwise, decreased selection among haploids is favoured.
Thus, because altering haploid selection intensity affects the alleles that are inherited by daughters, mothers can favour slightly biased sex ratios. 
In addition, I found that stronger sex ratio biases can be favoured by paternal manipulations of the haploid `selective arena' because fathers are strongly selected to maximize their own siring success (above selection to equalize the sex ratio).

}

{\color{blue}
FROM THESIS:
Generally, any sex-linked gene that harbours genetic variation in haploid fitness should cause sex ratios to become biased. 
Sex ratio bias caused by pollen competition has previously been discussed in the context of Y-linked deleterious mutations, which are thought to build up after recombination suppression evolves \citep{Lloyd:1974tz,Stehlik:2005ul}. 
Sex ratios can also become biased due to meiotic drive;
in a classic paper, \citet{Hamilton:1967ts} showed that X- or Y-linked alleles that experience meiotic drive will bias sex ratios. 
He assumed that driving alleles are under directional selection and spread to fixation but such alleles can also be maintained at intermediate frequencies by selection \citep{Feldman:1989gm,Holman:2015en}. 
When sex ratios are biased, other loci are expected to evolve to restore equal sex ratios. 
Indeed, alleles that negate the effect of sex-linked meiotic drivers and restore equal sex ratios have been identified \citep{Stalker:1961th,Smith:1975ft}. 
A similar process occurs with cytoplasmic male sterility alleles (that cause biased sex ratios) and nuclear `restorer' genotypes \citep{Frank:1989vl}. 

Several aspects of the relationship between haploid selection (e.g., pollen or sperm competition) and sex ratios remain to be explored. 
For example, new sex-determining systems (particularly transitions between male and female heterogamety) can be favoured in order to restore equal sex ratios in populations that have a sex ratio bias \citep{Bull:1983vi,Kozielska:2010vm,Ubeda:2015fx}.
Based on the results of Chapter \ref{ch:sexchromosomes}, we would expect that sex ratio biases would occur via associations between sex-determining loci and loci that experience haploid selection. 
However, these associations should also select against transitions between sex-determining systems, as has been found with sexually antagonistic selection \citep{vanDoorn:2007eu,vanDoorn:2010hu}. 
It is not clear how the spread of new sex determination systems would be influenced by the combination of sex ratio biases and favourable associations between haploid selected loci and sex-determining regions. 
Finally, \citet{Hamilton:1967ts} pointed out that biased sex ratios can affect population size because the number of offspring in each generation is typically determined by the number of females. 
Population density can, in turn, affect the intensity of pollen/sperm competition in future generations because fewer males are available to donate pollen/sperm in a particular area. 
Thus, a feedback could occur between population densities and haploid selection, which has not yet been investigated. 
}




\bibliographystyle{amnatnat.bst}
\bibliography{sex_chromosomes.bib}

\end{document}



