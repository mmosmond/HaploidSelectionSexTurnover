 \documentclass[12pt,letterpaper]{article}
\usepackage{fontenc}
\usepackage[english]{babel}

\usepackage{fullpage}

\usepackage[pdftex]{graphicx}

\begin{document}

\thispagestyle{empty}

\noindent Michael F.\ Scott\\
UCL Genetics Institute\\
Department of Genetics, Evolution \& Environment\\
University College London \\
2nd Floor, Darwin Bldg., Gower St., London, WC1E 6BT, UK\\
 m.f.scott@ucl.ac.uk\\
\smash{\hspace{12cm}\includegraphics[scale=0.5]{UCL.png}}

%\noindent Matthew M.\ Osmond\\
%Department of Zoology\\
%University of British Columbia\\
%6270 University Blvd. \\
%Vancouver, BC V6T 1Z4, Canada\\
%mmosmond@zoology.ubc.ca\\
%\smash{\hspace{11.5cm}\includegraphics[scale=0.25]{IMAGES/UBC.png}}

\noindent Dear Editor, \hfill \today \\

Please find the attached manuscript, entitled ``Haploid selection, sex ratio bias, and transitions between sex-determination systems'', which we submit to you for consideration for publication in \textit{PLoS Biology}. Laurence Hurst and Nick Barton would be well qualified Academic Editors for this manuscript. 
\\

In this manuscript, we develop and thoroughly analyse models to show what conditions favour the evolution of new sex-determination systems. Evolutionary transitions between sex-determination systems are an active research focus due to the surprising diversity and lability of sex determination systems, which is becoming increasingly apparent (Bachtrog et al. 2014, \textit{PLoS Biology}). Transitions in such a fundamental trait of large effect warrant evolutionary explanation; our results suggest several new scenarios under which new sex-determining systems are favoured, which could help to explain why the evolution of sex-determining systems is so dynamic.  \\

Previous modelling studies, including Muralidhar and Veller (2018, \textit{Nature Ecology and Evolution}) and van Doorn and Kirkpatrick (2007, Nature; 2010, Genetics), have shown that a new sex determining allele can spread if it is more tightly linked to a sexually-antagonistic locus. Our work finds several surprising results in light of previous results. First, when the old sex determining allele is already tightly linked to a sexually-antagonistic locus, more loosely linked sex determining factors can invade under some circumstances, offering a new potential explanation for sex chromosome turnover. Second, when we include selection in the haploid phase (among gametes or gametophytes, including meiotic drive), entirely new predictions are generated for the spread of new sex determining factors.
In particular, selective differences between male and female diploids are no longer necessary;  conflicts between haploid and diploid selection can instead drive the invasion of new sex chromosomes. Interestingly, new sex chromosomes can spread under conditions where they balance the sex ratio but also under conditions where they lead to more skewed sex ratios. Indeed, transitions during which sex ratio biases increase or decrease are equally likely to evolve under a wide range of conditions. We conclude that haploid selection should be considered as a pivotal factor driving transitions between sex-determining systems. \\

%Previous modelling studies include Muralidhar and Veller (2018, \textit{Nature Ecology and Evolution}), and van Doorn and Kirkpatrick (2007, \textit{Nature}; 2010, \textit{Genetics}), which have shown that a new sex determining allele can be favoured if it arises in linkage with sexually-antagonistic loci. 
%One novel feature of our models is that we explicitly consider selection on loci that are very tightly linked to the ancestral sex-determining region.
%Surprisingly, under various forms of selection, the unusual inheritance pattern of sex-linked loci can give rise to cases where the ancestral-X carries suboptimal alleles for females and/or the ancestral-Y carries female-beneficial alleles. 
%This, in turn, can favour new sex-determining alleles that have weaker linkage with the loci under selection, which has not been predicted by previous theory. \\

%It has been suggested that selection to balance the sex ratio is a dominant force driving transitions between sex-determining systems. 
%Our models include haploid selection (meiotic drive or gametic competition), which can cause zygotic-sex ratios to become biased. 
%In addition, haploid selection is usually sex-specific (occurring in either males or females only). 
%Consequently, we find that haploid selection can cause transitions analogous to those caused by sexually-antagonistic selection, eliminating the need for differences in selection between male and female diploids. 
%Unexpectedly, we do not find that selection to balance the sex ratio is overwhelming; transitions during which sex ratio biases increase or decrease are equally likely to evolve under a wide range of conditions. 
%Furthermore, we find that looser linkage with the sex-determining region more often evolves when there is haploid selection, increasing the potential for lability in sex-determination. 
%We conclude that haploid selection should be considered as a pivotal factor driving transitions between sex-determining systems. \\

Overall, we have developed an extensive set of models that explore the forces driving transitions between sex determination systems, an important aspect of diversity. 
Of particular interest, we find several results (e.g., looser linkage and sex ratio biases evolving) that are surprising, given previous theory. 
%Therefore, our models predict loci in previously unexpected genomic locations and/or experiencing various types of selection (including haploid selection) can now be implicated as drivers of transitions between sex-determining systems. 
We hope you agree that this work is of broad interest to the readership of \textit{PLoS Biology} and we eagerly anticipate your response. \\

%findings;why important
%Sex determination is remarkably dynamic.
%The two main theories for why we observe transitions are selection to 1) equalise the sex ratio and 2) increase linkage with a sexually-antagonistic locus. 
%Here we combine and extend these theories with population genetic models that include both haploid and diploid selection while simultaneously allowing for tight linkage.
%We show that both haploid selection and tight linkage can drive previously unexpected transitions between sex-determining systems.
%In particular, both can select for new sex-determining loci that are \textit{less closely linked} to loci under selection. 
%In addition, haploid selection can cause transitions analogous to those caused by purely sexually-antagonistic selection, \textit{eliminating the need for differences in selection between male and female diploids}.
%Further, transitions involving haploid selection can be driven by sex ratio selection or \textit{cause sex ratio biases to evolve}. 
%We conclude that haploid selection should be considered as a pivotal factor driving transitions between sex-determining systems. 
%Our results suggest several new scenarios under which new sex-determining systems are favoured, which could help  explain why the evolution of sex-determining systems is so dynamic, and indicate that new sex-determining alleles can arise at previously unexpected genomic locations.
%\\

%Based upon their expertise in theoretical quantitative genetics in changing environments, we suggest Dr.\ Luis-Miguel Chevin$^1$, Dr.\ Richard Gomulkiewicz$^2$, and Dr.\ Reinhard B\"urger$^3$ as potential reviewers. 

%This work is original and is not under consideration for publication elsewhere. 
%This work is original, unpublished and is not under consideration for publication elsewhere. %H
%All authors have agreed to the submission of the paper to \textit{Evolution}. %H

%We hope you agree that this work is of broad interest to the readership of \textit{PLoS Biology} and we eagerly anticipate your response.\\ 

Sincerely, \\
\newline
\\
\indent Michael Scott, Matthew Osmond, Sarah Otto

%reviewer emails
%\small{$^1$ luis-miguel.chevin@cefe.cnrs.fr;  $^2$ gomulki@wsu.edu; $^3$ reinhard.buerger@univie.ac.at}

\end{document}