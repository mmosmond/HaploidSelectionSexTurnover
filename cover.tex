 \documentclass[12pt,letterpaper]{article}
\usepackage{fontenc}
\usepackage[english]{babel}

\usepackage{fullpage}

\usepackage[pdftex]{graphicx}

\begin{document}

\thispagestyle{empty}

\noindent Michael F.\ Scott\\
Genetics, Evolution \& Environment\\
University College London Genetics Institute\\
2nd Floor, Darwin Bldg., Gower St., London, WC1E 6BT, UK\\
 m.f.scott@ucl.ac.uk\\
\smash{\hspace{12cm}\includegraphics[scale=0.5]{UCL.png}}

%\noindent Matthew M.\ Osmond\\
%Department of Zoology\\
%University of British Columbia\\
%6270 University Blvd. \\
%Vancouver, BC V6T 1Z4, Canada\\
%mmosmond@zoology.ubc.ca\\
%\smash{\hspace{11.5cm}\includegraphics[scale=0.25]{IMAGES/UBC.png}}

\noindent Dear Editor, \hfill \today \\

Please find the attached manuscript, entitled ``Haploid selection, sex ratio bias, and transitions between sex-determination systems'', which we submit to you for consideration for publication in \textit{PLoS Biology}.
\\

%findings;why important
Sex determination is remarkably dynamic.
The two main theories for why we observe transitions are selection to 1) equalise the sex ratio and 2) increase linkage with a sexually-antagonistic locus. 
Here we combine and extend these theories with population genetic models that include both haploid and diploid selection while simultaneously allowing for tight linkage.
We show that both haploid selection and tight linkage can drive previously unexpected transitions between sex-determining systems.
In particular, both can select for new sex-determining loci that are \textit{less closely linked} to loci under selection. 
In addition, haploid selection can cause transitions analogous to those caused by purely sexually-antagonistic selection, \textit{eliminating the need for differences in selection between male and female diploids}.
Further, transitions involving haploid selection can be driven by sex ratio selection or \textit{cause sex ratio biases to evolve}. 
We conclude that haploid selection should be considered as a pivotal factor driving transitions between sex-determining systems. 
Our results suggest several new scenarios under which new sex-determining systems are favoured, which could help  explain why the evolution of sex-determining systems is so dynamic, and indicate that new sex-determining alleles can arise at previously unexpected genomic locations.
\\

%Based upon their expertise in theoretical quantitative genetics in changing environments, we suggest Dr.\ Luis-Miguel Chevin$^1$, Dr.\ Richard Gomulkiewicz$^2$, and Dr.\ Reinhard B\"urger$^3$ as potential reviewers. 

%This work is original and is not under consideration for publication elsewhere. 
%This work is original, unpublished and is not under consideration for publication elsewhere. %H
%All authors have agreed to the submission of the paper to \textit{Evolution}. %H

We hope you agree that this work is of broad interest to the readership of \textit{PLoS Biology} and we eagerly anticipate your response.\\ 

Sincerely,

[signature]\\
%\includegraphics[height=1.5cm]{IMAGES/OsmondSignature.pdf}
\indent Michael Scott, Matthew Osmond, Sarah Otto

%reviewer emails
%\small{$^1$ luis-miguel.chevin@cefe.cnrs.fr;  $^2$ gomulki@wsu.edu; $^3$ reinhard.buerger@univie.ac.at}

\end{document}